\documentclass[12pt]{article}
\usepackage[letterpaper]{geometry}
\usepackage{setspace}
\geometry{top=0.9in, bottom=0.9in, left=0.9in, right=0.9in}
\doublespacing

\makeatletter
\def\@maketitle{%
  \newpage
  \null
  \vskip 2em%
  \begin{center}%
    {\large
        \begin{tabular}[t]{c}%
        \@author
        \end{tabular}\par}%
  \let \footnote \thanks
    {\LARGE \@title \par}%
    %{\large \@date}%
  \end{center}%
  \par
  \vskip 1.5em}
\makeatother

\author{Mark Kim}
\title{Essay 1}

\begin{document}
\maketitle

The explosion of deep learning has made technologies such as OpenAI's ChatGPT
and Google's Gemini household names.  The excitement that follows such
attention, however, engenders a narrow-minded focus, constraining what is
possible only to what has become available.  Despite their ability
to learn to complete complex specialized tasks and mimic human written
composition, such models can only give shallow understanding on how our minds
operate. We may be able to infer some properties of low-level brain function
from these neural networks, but insight into higher cognition eludes us.  These higher
functions of cognition are what pique my curiosity.  In particular, I wish to
delve into the inner workings of motivation and long-term passions, but studying
these things directly is untenable as the factors involved with any individual's
motivations are not only intractably numerous, but also may not be feasibly
measurable.  My ultimate goal is to instead investigate and model biologically
plausible neural networks as a proxy of brain function.

State-of-the-art deep learning algorithms today rely on mechanisms that are
biologically implausible.  They depend on gradient back-propagation, which
computes the gradient of an objective function with respect to the weights of a
neural network.  Such back-propagation raises problematic issues that
demonstrate the improbability of such a process in biology.  First,
back-propagation is a purely linear computation, while biological neurons apply
both linear and non-linear operations. Credit assignment, which is the act of
determining the influence that an action taken will have on future rewards, in
such a paradigm, would require precise knowledge of the gradient in both
directions and exact symmetry for the weights.  Futhermore, artificial neurons
communicate by continuous values, while their biological counterparts
communicate through action potentials, which are binary in nature.  Finally,
most deep learning models have discrete training and prediction stages.

Although there has been research towards biologically plausible machine
learning, this area of exploration is still in its infancy.  There are many
avenues of study in this field with a plethora of opportunities to expand upon
already completed research.  Much of the current research attempts to address
one or more of the problems associated with gradient back-propagation.
% the previous sentence is slightly off-kilter.
One
area of research that is of particular interest to me is continual learning, which is a
hallmark of human intelligence.  Recent work into continual learning use a
combination of multiple techniques to allow artificial neural networks to
consolidate synapses to mitigate forgetting and strengthen connections between
contextual information.  Nevertheless, the research still relied on
back-propagation of fully connected networks.  Brain synapses are unidirectional
with physically distinct feed-forward and feedback connections. It is also
believed that the brain is capable of localized learning. Investigating neuron
architectures with these features in continual learning is an exciting prospect
of study for me.

Researching novel neural networks that more closely mimic biological systems could
help uncover some of the mechanisms of higher cognition.  Likewise, creating a model of
cognition that closely simulates biological brains could unlock new insight into
neurological diseases, mental illness and wellness, and learning.  By pursuing a
doctoral degree in Computational Cognitive Science, I will be among
multidisciplinary experts that would be best equipped to assist me with my
research interests.  Although my long-term goals align more with affective
science, affect is intimately linked with cognition and behavior.  It is my hope
that through my journey I will be making incremental discoveries towards
unearthing a greater understanding of affect, cognition, and the
interrelationship between both.

\end{document}