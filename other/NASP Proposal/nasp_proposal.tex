\documentclass[12pt]{article}
\usepackage{mysty}
\usepackage[backend=biber]{biblatex}

\addbibresource{nasp_proposal.bib}

\title{Metacognition in Computer Science Learning: Perception vs. Reality}
\date{\today}

\begin{document}
\maketitle
\section{Abstract}
Students majoring in computer science often have little to no computing
experience before entering college and this can affect students' confidence and
performance throughout their education~\cite{alvarado}.  To mitigate this
obstacle, we attempt to boost students’ confidence, reduce learning loss, and
provide foundational support by offering a four day online winter term bootcamp
as a bridge between their introductory and intermediate computer science
courses.  This is a qualitative case study examining the experiences of Computer
Science students that participated with an emphasis on self-efficacy and the
importance of providing scaffolding to ensure that they stay within the zone of
proximal development~\cite{vygotsky, wood1975, wood1976}.  Although this is a
study of college students, results may translate to secondary schools or other
settings where introductory computing classes are being offered.

\section{Introduction}
% With the introduction and rapid growth of artificial intelligence in recent
% years, computer science has increasingly become a driver of innovation in the US
% economy.  In 2023, there were over 350,000 open computing jobs nationwide but
% only 90,000 students who received a computer science bachelor's
% degree~\cite{code}.  Liu, et. al. (2024) suggests that access to CS courses in
% secondary school increases the likelihood that students will declare a CS major
% by 10\% and completing a CS bachelor degree by 5\%~\cite{liu}, yet CS education
% has still not become an integral part of our eduction system with only 29 states
% adopting policies that provide CS course access to all high school students as
% of 2023~\cite{code}.  Because of this gap in secondary education, students
% entering college are often missing the foundational knowledge necessary for
% self-guided study.
% Additionally, 
CS education is highly sequential, where courses rely heavily on
concepts learned in prerequisite courses.  A course in Data Structures has
substantial reliance on Object Oriented Programming concepts, which, in turn,
relies on understanding of Control Structures.  Beyond learning important
abstract concepts, students taking these introductory courses must also learn
concrete programming language specific commands, syntax, and conventions.  The
amount of information that CS students must assimilate in a short period
often requires instructors to abridge the material that they cover and focus on
abstract concepts, leaving students with little applied understanding.  This is
particularly concerning for introductory CS courses where students are expected
to build upon these foundations.

This study aims to explore students' subjective experiences and perceptions of
learning computer science in a voluntary intersession bootcamp-style review and
preparation clinic.  It seeks to understand how students navigate the learning
process, the challenges they face, and their perceived proficiency.  By
investigating these experiences, we hope to provide insights that can assist
instructors with supporting student metacognition by providing them the
scaffolding to practice an learn at any level.

\section{Methods}
First year college students majoring in Computer Science participated
in a voluntary four day winter intersession bootcamp-style review and
preparation clinic between their first and second semesters at San Francisco
State University. The participants received a total of twenty hours of
instruction and lab exercises divided equally into the four days.  Each
participant was asked to complete daily open-ended surveys administered via
Google Forms about the efficacy, enjoyment, learning outcomes, and self-assessed
proficiency of each day's activities. Notes from the instructor were collected
on the content of instruction, observations from each session, and reflections
on student engagement and proficiency.

\section{Results}
Data collection is complete, and qualitative and quantitative analysis of
results are underway, with expected completion by August 15, 2024.  Twenty-one
students participated in the clinic and a total of 42 surveys were collected.
As proposed by Braun and Clarke~\cite{braun}, a thematic analysis of the student
survey responses shall be conducted.  The instructor's notes will be paired with
the thematic analysis to provide insight of either confirmation or contradiction
between the students' self-reported experiences and the instructor's
observations.

Initial analysis of the survey responses suggests that students reported a high
level of enjoyment and extracted learning benefit.  Surprisingly, the
instructor's notes do not suggest significant enjoyment from the students, as
student engagement was found to be lacking.  Students reported confidence in
their understanding of some basic programming concepts, but instructor
observations indicate that this confidence may be misplaced.

\section{Discussion}
While metacognition has been extensively researched and is a crucial aspect of
learning, studying any disparities between computer science students' perceived
understanding and actual performance has not seen much, if any, attention.
Moreover, the instructor's observations of student engagement also did not seem
to match the students' self-reported enjoyment and extracted learning benefit.
This study aims to explore the relationship between students' self-assessed
experiences and instructor-observed performance and engagement.  Initial results
from the bootcamp clinic suggest that students may not fully understand their own
proficiency in programming concepts, which could have implications for future
instruction.  Likewise, the instructor may not be fully understanding the level
of student engagement and enjoyment, which could impact the design of future
learning experiences.  An expanded discussion, limitations, and implications
about these complex relationships will be provided in the forthcoming
presentation to help inform the practices of school psychologists.

\printbibliography
\end{document}