\documentclass[12pt]{article}
\usepackage[letterpaper]{geometry}
\usepackage{setspace}
\geometry{top=.9in, bottom=.9in, left=.9in, right=.9in}
\doublespacing

\makeatletter
\def\@maketitle{%
  \newpage
  \null
  \vskip 2em%
  \begin{center}%
    {\large
        \begin{tabular}[t]{c}%
        \@author
        \end{tabular}\par}%
  \let \footnote \thanks
    {\LARGE \@title \par}%
    %{\large \@date}%
  \end{center}%
  \par
  \vskip 1.5em}
\makeatother

\title{Essay 3}
\author{Mark Kim}

\begin{document}
\maketitle

When I re-entered college I never expected to have an interest in serving a
college or university as a member of the faculty.  In fact, my return to school was
predicated entirely on a business idea and personal interest based in fueling
people's career decisions from latent passions that may be found through data
mining.  It wasn't until I stumbled into a student led supplemental instruction
course in introductory programming that a spark of interest in academia as an
instructor was kindled within me.  As I became more integrated with the
amazingly inclusive and supportive academic community at San Francisco State
University, I could increasingly imagine myself contributing to it as many of my
faculty mentors do.

The supplemental instruction program that I so fondly participated in as a
pupil unveiled the possibility that I, even as an undergraduate student, could
also lead and facilitate such a course.  Before this, I assumed that my
knowledge and past endeavors in the field of Computer Science and Mathematics were
insufficient to be of value to other students.  Still uncertain of my abilities,
I found a position that I thought I could manage: grading math homework
assignments.  In doing so, I also learned that my work served to refresh and
solidify my understanding from the courses that I was grading for.

During my stint as a grader, and in response to COVID-19 pandemic lockdowns, I
was provided the chance to assist a math education professor do academic
literature review on remote learning pedagogy.  Both fascinating and eye-opening
to me, this study introduced me to the complexities and challenges of
educating students remotely.  As I read and reviewed more articles and research
papers, I contemplated the hypothetical scenario of how I would teach students
in Math or Computer Science if I were given the opportunity.  Likewise, my
supervising professor ruminated over the underdeveloped state of research into
Computer Science education when contrasted against Mathematics education.  Since
I had one foot firmly in each discipline, I experienced this disparity directly.

Shortly after completing my brief encounter with Mathematics education research,
I was awarded the opportunity to become a supplemental instruction facilitator
in Calculus.  Armed with my prior experiences, the prospect of applying my newly
gained knowledge thrilled me.  I was determined to apply what I learned and attempt to
deliver an experience that not only inspired students, but also significantly
improved their learning outcomes.  Then after two semesters, I able to move
laterally into Computer Science, where I believed that I could provide even greater
impact.  My time in this program was deeply fulfilling, which
resulted in my continued participation in this capacity through my completion of
my undergraduate degree.

The intrinsic rewards of teaching and witnessing student success drove me to
seek other ways I could contribute to the community of students.  As a result, I
accepted the role as president of the student chapter of the Association of
Computing Machinery to spearhead a post-pandemic revival.  Furthermore, I took
the program lead position for AI-STAARS, a scholarship program aimed at under-represented
minorities in the field of Computer Science.  The responsibilities of this role
were far more extensive than as a facilitator. In addition to preparing and
executing weekly lesson plans, which was familiar to me from my other role as
facilitator, I was tasked to lead a week long winter coding bootcamp and an
entire ten week summer pathways Artificial Intelligence internship.

All these experiences exposed me to a diverse collection of students, all of
whom carried fascinating stories.  I believe this exposure has initiated my
preparation for serving a diverse student body and I am eager to continue to
improve.  I am incredibly grateful of having been given the chance to get to
know these students and am looking forward to meeting and becoming acquainted
with more.
\end{document}