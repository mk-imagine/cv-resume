\documentclass[hidelinks, 10.5pt]{article}
\usepackage{amssymb, multicol, graphicx, hyperref}
\usepackage[T1]{fontenc}
\usepackage{fontspec}
\hypersetup{
	pdftitle={Resume},
	pdfsubject={Resume},
	pdfauthor={Mark Kim}
}

% NEEDS UPDATING TO MATCH NO BORDER VERSION

\usepackage[dvipsnames]{xcolor}

\usepackage[letterpaper,
            heightrounded,
            left=0.65in,
            right=0.7in,
            top=0.7in,
            bottom=0.7in]{geometry}

\usepackage{tikz}
\usetikzlibrary{positioning}
\usetikzlibrary{patterns}
\usetikzlibrary{calc}

\usepackage{ebgaramond}

\graphicspath{ {./images/} }

\DeclareSymbolFont{numbers}{T1}{EBGaramond-LF}{m}{n}
\SetSymbolFont{numbers}{bold}{T1}{EBGaramond-LF}{bx}{n}
\DeclareMathSymbol{0}\mathalpha{numbers}{"30}
\DeclareMathSymbol{1}\mathalpha{numbers}{"31}
\DeclareMathSymbol{2}\mathalpha{numbers}{"32}
\DeclareMathSymbol{3}\mathalpha{numbers}{"33}
\DeclareMathSymbol{4}\mathalpha{numbers}{"34}
\DeclareMathSymbol{5}\mathalpha{numbers}{"35}
\DeclareMathSymbol{6}\mathalpha{numbers}{"36}
\DeclareMathSymbol{7}\mathalpha{numbers}{"37}
\DeclareMathSymbol{8}\mathalpha{numbers}{"38}
\DeclareMathSymbol{9}\mathalpha{numbers}{"39}

\setlength\columnsep{1pt}

%%%%%%%%%%%%%%%%%%%%%%%%%%%%%%%%%%%%%%%%%%%%%%%%%%%%%%%%%%%%%%%%%%%%%%%%%%%%%%%%%%%%%%%%%%%%%%%%%%%%%%%%%%%%%%%%%%%%%%%%%%%
% Element Spacing
%%%%%%%%%%%%%%%%%%%%%%%%%%%%%%%%%%%%%%%%%%%%%%%%%%%%%%%%%%%%%%%%%%%%%%%%%%%%%%%%%%%%%%%%%%%%%%%%%%%%%%%%%%%%%%%%%%%%%%%%%%%
\def\contentwidth{0.9\linewidth}    % width of content (under the section names)
\def\contentblockspacing{2.5mm}     % spacing between content blocks
\def\contentheaderspacing{1mm}      % spacing between content header and content title/body
\def\sectionspacing{8mm}            % spacing between last element in contents and next section
\def\sectiontocontentspacing{4mm}   % spacing between the section titles and the first content element

%%%%%%%%%%%%%%%%%%%%%%%%%%%%%%%%%%%%%%%%%%%%%%%%%%%%%%%%%%%%%%%%%%%%%%%%%%%%%%%%%%%%%%%%%%%%%%%%%%%%%%%%%%%%%%%%%%%%%%%%%%%
% Section Settings
%%%%%%%%%%%%%%%%%%%%%%%%%%%%%%%%%%%%%%%%%%%%%%%%%%%%%%%%%%%%%%%%%%%%%%%%%%%%%%%%%%%%%%%%%%%%%%%%%%%%%%%%%%%%%%%%%%%%%%%%%%%
\makeatletter
\renewcommand{\section}[1]{
    {\fontsize{14}{14}\selectfont \textsc{\textbf{\ \ #1\ \ }}}\hrulefill
}
\makeatother

%%%%%%%%%%%%%%%%%%%%%%%%%%%%%%%%%%%%%%%%%%%%%%%%%%%%%%%%%%%%%%%%%%%%%%%%%%%%%%%%%%%%%%%%%%%%%%%%%%%%%%%%%%%%%%%%%%%%%%%%%%%
% Personal Information Variables
%%%%%%%%%%%%%%%%%%%%%%%%%%%%%%%%%%%%%%%%%%%%%%%%%%%%%%%%%%%%%%%%%%%%%%%%%%%%%%%%%%%%%%%%%%%%%%%%%%%%%%%%%%%%%%%%%%%%%%%%%%%
\def\fullname{Mark S. Kim}                              % Full Name
\def\phonenumber{5104997953}                            % Phone Number (for tel link)
\def\phonenumberdisplay{510.499.7953}                   % Displayed Phone Number
\def\email{mkim797@yahoo.com}                           % Email Address
\def\linkedinname{mkim797}                              % LinkedIn Username
\def\linkedinurl{https://www.linkedin.com/in/mkim797}   % LinkedIn Profile URL
\def\githubname{mk-imagine}                             % GitHub Username
\def\githuburl{https://github.com/mk-imagine/}          % GitHub Profile URL

\begin{document}

\pagestyle{empty}

\begin{tikzpicture}[overlay,remember picture]
    \node (address) at ($ (current page.north east) + (-4.45,-0.98) $)
    {{
                \href{tel:\phonenumber}{\raisebox{-1mm}
                    {\includegraphics[scale=0.052]{phone.png}}\,\ \phonenumberdisplay}
                \enspace {\scriptsize{$\blacklozenge$}} \enspace
                \href{mailto:\email}
                {\raisebox{-1mm}
                    {\includegraphics[scale=0.055]{env.png}}
                    \email}}};
    \node (address) at ($ (current page.north east) + (-3.7,-1.65) $)
    {\href{\linkedinurl}
        {\raisebox{-0.9mm}
            {\includegraphics[scale=0.055]{li.png}}
            \hspace{0.7mm} \linkedinname}
        \enspace {\scriptsize{$\blacklozenge$}} \enspace
        \href{\githuburl}
        {\raisebox{-0.9mm}
            {\includegraphics[scale=0.055]{gh.png}}
            \hspace{0.5mm} \githubname}};
    \draw [line width=3.5pt,rounded corners=4pt]
    ($ (current page.north west) + (1cm,-1.3cm) $)
    rectangle
    ($ (current page.south east) + (-1cm,1cm) $);
    \draw [fill=white,draw=none]
    ($ (current page.north west) + (2cm,-0.5cm) $)
    rectangle
    ($ (current page.north west) + (8.85cm,-2cm) $);
    \node [draw=none] at ($ (current page.north west) + (5.4cm,-1.25cm) $)
    {\fontsize{33}{21}\selectfont \textbf{\fullname}};
\end{tikzpicture}

%%%%%%%%%%%%%%%%%%%%%%%%%%%%%%%%%%%%%%%%%%%%%%%%%%%%%
% EDUCATION
%%%%%%%%%%%%%%%%%%%%%%%%%%%%%%%%%%%%%%%%%%%%%%%%%%%%%

{\centering
\vspace{5mm}

\section{Education}

\vspace{\sectiontocontentspacing}

\begin{minipage}[ct]{0.9\linewidth}
    {\textsc{San Francisco State University}}\hfill 2023 to Present\\
    \emph{Data Science and Artificial Intelligence, M.S.}\\
    GPA: $4.00$\\
    Anticipated Graduation: May 2025
\end{minipage}

\vspace{\contentblockspacing}

\begin{minipage}[ct]{0.9\linewidth}
    {\textsc{San Francisco State University}}\hfill 2018 to 2023\\
    \emph{Computer Science, B.S.}\\
    GPA: $4.00$\\
    \emph{Mathematics: Advanced Studies, B.A.}\\
    GPA: $3.98$
\end{minipage}

% \vspace{3mm}

% \begin{minipage}[ct]{0.9\linewidth}
% {\textsc{University of California at Davis}}\hfill 1995 to 2000\\
% \emph{Uncompleted Genetics B.S.}
% \end{minipage}

\vspace{\sectionspacing}

%%%%%%%%%%%%%%%%%%%%%%%%%%%%%%%%%%%%%%%%%%%%%%%%%%%%%
% RESEARCH EXPERIENCE
%%%%%%%%%%%%%%%%%%%%%%%%%%%%%%%%%%%%%%%%%%%%%%%%%%%%%

\section{Research Experience}

\vspace{\sectiontocontentspacing}

\begin{minipage}[ct]{\contentwidth}
    \textbf{Graduate Researcher}\hfill Aug 2023 to Present\\
    \textsc{San Francisco State University} -- San Francisco, CA\\
    Principal Investigator: Dr. Hui Yang
    \vspace{\contentheaderspacing}\\
    {\textit{AdvisingGPT: Foundation Models for Student Advising}}\\
    An exploration of using proprietary and open-source foundation models to provide automated course equivalency evaluation and
    personalized program roadmaps to maximize student success rates. The techniques employed include: prompt engineering, in-context
    learning, and instruction fine-tuning of foundation and embedding models; document-level embeddings search and ranking; and
    retrieval augmented generation.
\end{minipage}

\vspace{\contentblockspacing}

\begin{minipage}[ct]{\contentwidth}
    \textbf{Graduate Researcher/Program Lead}\hfill Jan 2023 to Present\\
    \textsc{San Francisco State University} -- San Francisco, CA\\
    Principal Investigator: Dr. Anagha Kulkarni
    \vspace{\contentheaderspacing}\\
    {\textit{Artificial Intelligence Scholarships that Improve Academic Achievement, Retention, and Career Success (AI-STAARS)}}\\
    % Lead a three pronged support system to improve student success by reinforcing
    % foundational knowledge, providing intensive advising, and engaging students in a
    % winter and summer accelerated pathways Machine Learning and AI program.  Besides providing these
    % activities, the program was
    An examination of methods to improve students' sense of belonging and identity with the field of Computer Science,
    and the effect it may have on students' retention and success.  Qualitative research on metacognition through a comparative
    analysis of in-class assessed performance against students' perceived understanding.  Data is collected through interviews, surveys,
    and observations of students during discussions.
\end{minipage}

\vspace{\contentblockspacing}

\begin{minipage}[ct]{\contentwidth}
    \textbf{Research Engineering Intern}\hfill Jun 2022 to Aug 2022\\
    \textsc{Cofense Inc.} -- Leesburg, VA\\
    Research Supervisor: Chip McSweeney, Senior Research Engineer
    \vspace{\contentheaderspacing}\\
    {\textit{Phishing Emails: Clustering and Analysis}}\\
    An investigation of clustering for the early detection and categorization of phishing emails with an emphasis on computational speed
    and performance.  Python C extensions that parse and analyze emails were restructured and optimized, which reduced memory usage by
    $95$\% and increased data utilization by $5$\%. Similarly, development and validation of thead-based and process-based asynchronous
    parallelization of the Python code base reduced processing time by $80$\%. Tradeoffs between dimensional reduction (PCA) and
    maintaining data precision were examined and analyzed.
    % \makeatletter
    % \let\orig@listi\@listi
    % \def\@listi{\orig@listi\topsep=0.25\baselineskip}
    % \makeatother
    % \begin{itemize}
    % 	\setlength\itemsep{-0.25em}
    % 	\item Investigation of clustering for early detection and
    % 	categorization of phishing emails with an emphasis on computational speed and
    % 	performance.
    % 	\item Restructure and optimize Python C extensions that parse and
    % 	analyze emails.  Reduced memory usage by 95\% and increased data
    % 	utilization by 5\%.
    % 	\item Develop and validate thread-based and process-based
    % 	parallelization (asynchronous).  Reduced total processing time by 80\%.
    % 	\item Examine and analyze tradeoffs between dimensional reduction (PCA) and
    % 	maintaining data precision.
    % \end{itemize}
\end{minipage}

\vspace{\contentblockspacing}

% \begin{minipage}[ct]{\contentwidth}
%     \textbf{Independent Study and Research Literature Review}\hfill Jan 2022 to May 2022\\
%     \textsc{San Francisco State University Department of Mathematics} -- San Francisco, CA\\
%     Research Supervisor: Dr. Serkan Hosten
%     \vspace{\contentheaderspacing}\\
%     {\textit{Graphical Models for Brain Networks}}\\
%     Studied and explored \emph{Graphical Models} by Steffen Lauritzen which
%     culminated in an examination of the research completed by Ranciati, Saverio et
%     al. in ``Fused graphical lasso for brain networks with symmetries.''
%     % \makeatletter
%     % \let\orig@listi\@listi
%     % \def\@listi{\orig@listi\topsep=0.25\baselineskip}
%     % \makeatother
%     % \begin{itemize}
%     % 	\setlength\itemsep{-0.25em}
%     % 	\item Study and exploration of \emph{Graphical Models} by Steffen
%     % 	Lauritzen and ``Fused graphical lasso for brain networks with
%     % 	symmetries'' by Ranciati, Saverio et al.
%     % \end{itemize}
% \end{minipage}

% \vspace{\contentblockspacing}

\begin{minipage}[ct]{\contentwidth}
    \textbf{NSF REU Scholar and Researcher}\hfill Jun 2021 to Aug 2021\\
    \textsc{University of Houston} -- Houston, TX\\
    Funding by the National Science Foundation\\
    Principal Investigator: Dr. Ernst Leiss\\
    Research Supervisor: Dr. Ionnis Pavlidis
    \vspace{\contentheaderspacing}\\
    {\textit{Frontiers of Data-Driven Computing REU}}\\
    Developed and implemented multi-threaded retrieval algorithms for over 10 million records of affective research data (documents and
    authors) from Scopus, PubMed, and Web of Science. Performed exploratory clustering and co-occurrence matrix analysis of retrieved
    data to facilitate the investigation of a quantitative history of affective research.
    % \makeatletter
    % \let\orig@listi\@listi
    % \def\@listi{\orig@listi\topsep=0.25\baselineskip}
    % \makeatother
    % \begin{itemize}
    % 	\setlength\itemsep{-0.25em}
    % 	\item Develop and implement retrieval algorithms for affective research
    % 	data (documents and authors) from Scopus, PubMed, and Web of Science.
    % 	\item Exploratory clustering and co-occurrence matrix analysis of
    % 	retrieved data for a quantitative history of affective research.
    % \end{itemize}
\end{minipage}

\vspace{\sectionspacing}

% \begin{minipage}[ct]{\contentwidth}
%     \textbf{Innovation and Entrepreneurship Fellow}\hfill Sept 2020 to May 2021\\
%     \textsc{San Francisco State University Lam Family College of Business} -- San Francisco, CA\\
%     Faculty Director: Dr. Sybil Yang
%     \vspace{\contentheaderspacing}\\
%     {\textit{COB Innovation and Entrepreneurship Fellowship}}\\
%     Collaborated with co-founders to design and develop a software-based test
%     preparation platform for disenfranchised students.
%     % \makeatletter
%     % \let\orig@listi\@listi
%     % \def\@listi{\orig@listi\topsep=0.25\baselineskip}
%     % \makeatother
%     % \begin{itemize}
%     % 	\setlength\itemsep{-0.25em}
%     % 	\item Cross-functional collaboration with co-founders to design and develop test preparation platform for disenfranchised students.
%     % \end{itemize}
% \end{minipage}

% \vspace{\contentblockspacing}

\begin{minipage}[ct]{\contentwidth}
    \textbf{Undergraduate Research Assistant}\hfill Apr 2020 to Jun 2020\\
    \textsc{San Francisco State University} -- San Francisco, CA\\
    Research Supervisor: Dr. Shandy Hauk
    \vspace{\contentheaderspacing}\\
    {\textit{Remote Instruction Pedagogy in Mathematics}}\\
    Provided an academic literature review of research in pedagogical best practices for remote instruction.  This review was to inform
    new research in remote instruction in response to the COVID-$19$ pandemic.
    % \makeatletter
    % \let\orig@listi\@listi
    % \def\@listi{\orig@listi\topsep=0.25\baselineskip}
    % \makeatother
    % \begin{itemize}
    % 	\setlength\itemsep{-0.25em}
    % 	\item Academic literature review of research in pedagogical best practices for remote instruction.
    % \end{itemize}
\end{minipage}

\vspace{\contentblockspacing}

%%%%%%%%%%%%%%% SECOND PAGE %%%%%%%%%%%%%%%%%%
\begin{tikzpicture}[overlay,remember picture]
    \draw [line width=3.5pt,rounded corners=4pt,
    ]
    ($ (current page.north west) + (1cm,-1cm) $)
    rectangle
    ($ (current page.south east) + (-1cm,1cm) $);
\end{tikzpicture}

\begin{minipage}[ct]{\contentwidth}
    \textbf{Undergraduate Research Assistant}\hfill Nov 2019 to Jan 2020\\
    \textsc{San Francisco State University} -- San Francisco, CA\\
    Research Supervisor: Dr. Alexandra Piryatinska
    \vspace{\contentheaderspacing}\\
    {\textit{Change-point Analysis Algorithm Development}}\\
    Attended workshops in numerical methods and statistics theory in preparation for research in change-point analysis and algorithm
    development.  Studied completed change-point analysis research and began work on adapting existing Matlab code to Python.
    % \makeatletter
    % \let\orig@listi\@listi
    % \def\@listi{\orig@listi\topsep=0.25\baselineskip}
    % \makeatother
    % \begin{itemize}
    %     \setlength\itemsep{-0.25em}
    %     \item Workshops in numerical methods and statistics theory in
    %     preparation for research in change-point analysis and algorithm
    %     development.
    %     \item Study of completed change-point analysis research with the goal of
    %     adapting Matlab code to Python.
    % \end{itemize}
\end{minipage}

\vspace{\sectionspacing}

%%%%%%%%%%%%%%%%%%%%%%%%%%%%%%%%%%%%%%%%%%%%%%%%%%%%%
% Teaching and Advising
%%%%%%%%%%%%%%%%%%%%%%%%%%%%%%%%%%%%%%%%%%%%%%%%%%%%%

\section{Teaching \& Advising}

\vspace{\sectiontocontentspacing}

\begin{minipage}[ct]{\contentwidth}
    \textbf{Program Lead/Graduate Researcher}\hfill Jan 2023 to Present\\
    {\textsc{San Francisco State University} - San Francisco, CA}\\
    \textit{Artificial Intelligence Scholarships that Improve Academic Achievement, Retention, and Career Success (AI-STAARS)}
    \vspace{\contentheaderspacing}\\
    Develop and lead a three pronged support system to improve student success by reinforcing foundational knowledge, providing intensive
    advising, and engaging students with activities and exercises. This includes: a weekly intra-term lecture and workshop series; a weeklong
    programming foundations bootcamp to reduce learning loss; and a 10 week accelerated summer course in Machine Learning and
    Artificial Intelligence.
\end{minipage}

\vspace{\contentblockspacing}

\begin{minipage}[ct]{\contentwidth}
    \textbf{Mathematics Program Liaison}\hfill Jan 2024 to May 2024\\
    \textbf{Computer Science Program Liaison}\hfill Aug 2023 to Dec 2023\\
    {\textsc{San Francisco State University} - San Francisco, CA}\\
    \textit{Center for Science and Mathematics Education (CSME)}
    \vspace{\contentheaderspacing}\\
    Lead a team of undergraduate facilitators that teach and support students taking supplementary courses in Mathematics and Computer
    Science.
\end{minipage}

\vspace{\contentblockspacing}

\begin{minipage}[ct]{\contentwidth}
    \textbf{Facilitator}\hfill Jan 2021 to May 2023\\
    {\textsc{San Francisco State University} - San Francisco, CA}\\
    \textit{Center for Science and Mathematics Education (CSME)}
    \vspace{\contentheaderspacing}\\
    Develop lesson plans and lead supplementary courses in Mathematics and Computer Science.  These complementary courses deepen subject
    knowledge and improve student outcomes in their parent courses.
\end{minipage}

\vspace{\contentblockspacing}

\begin{minipage}[ct]{\contentwidth}
    \textbf{Undergraduate Teaching Assistant}\hfill Aug 2019 to Dec 2020\\
    {\textsc{San Francisco State University} - San Francisco, CA}\\
    \textit{Department of Mathematics}
\end{minipage}

\vspace{\sectionspacing}

%%%%%%%%%%%%%%%%%%%%%%%%%%%%%%%%%%%%%%%%%%%%%%%%%%%%%
% Conferences
%%%%%%%%%%%%%%%%%%%%%%%%%%%%%%%%%%%%%%%%%%%%%%%%%%%%%

{\fontsize{14}{14}\selectfont \textsc{\textbf{\ \ Conferences\ \ }}}\hrulefill

\vspace{\sectiontocontentspacing}

\begin{minipage}[ct]{\contentwidth}
    ``Metacognition in Computer Science Learning: Perception vs. Reality,'' National Association of School Psychologists Annual Convention.
    Seattle, WA, USA. Accepted and forthcoming, February 2025.
\end{minipage}

\vspace{\contentblockspacing}

\begin{minipage}[ct]{\contentwidth}
    ``Foundation Models for Course Equivalency Evaluation,'' IEEE International Conference on Data Mining.  Abu Dhabi, UAE.  Accepted and
    forthcoming, December 2024.
\end{minipage}

\vspace{\sectionspacing}

%%%%%%%%%%%%%%%%%%%%%%%%%%%%%%%%%%%%%%%%%%%%%%%%%%%%%
% HONORS & AWARDS
%%%%%%%%%%%%%%%%%%%%%%%%%%%%%%%%%%%%%%%%%%%%%%%%%%%%%

% \section{Honors \& Awards}

% \vspace{3mm}

% \begin{minipage}[ct]{0.9\linewidth}
%     First place final presentation -- Frontiers in Data-Driven Computing REU,
%     University of Houston\hfill Aug 2021

%     \vspace{1.5mm}

%     Finalist -- Entrepreneurship Symposium Innovation Pitch Competition, San
%     Francisco State University\hfill May 2021

%     \vspace{1.5mm}

%     Dean's List, San Francisco State University\hfill 2018, 2019, 2020, 2021, 2022, 2023
% \end{minipage}

% \vspace{\sectionspacing}

%%%%%%%%%%%%%%%%%%%%%%%%%%%%%%%%%%%%%%%%%%%%%%%%%%%%%
% LEADERSHIP
%%%%%%%%%%%%%%%%%%%%%%%%%%%%%%%%%%%%%%%%%%%%%%%%%%%%%

\section{Leadership}
\vspace{\sectiontocontentspacing}

\begin{minipage}[ct]{0.9\linewidth}
    \textsc{Association for Computing Machinery} (ACM), SFSU Student Chapter\hfill
    Sept 2019 to Present\\
    \emph{Graduate Mentor}, May 2024 to Present\\
    \emph{Treasurer}, May 2023 to May 2024\\
    \emph{President}, Jan 2022 to May 2023
\end{minipage}

\vspace{\contentblockspacing}

%%%%%%%%%%%%%%% THIRD PAGE %%%%%%%%%%%%%%%%%%
\begin{tikzpicture}[overlay,remember picture]
    \draw [line width=3.5pt,rounded corners=4pt,
    ]
    ($ (current page.north west) + (1cm,-1cm) $)
    rectangle
    ($ (current page.south east) + (-1cm,1cm) $);
\end{tikzpicture}
%%%%%%%%%%%%%%%%%%%%%%%%%%%%%%%%%%%%%%%%%%%%%
\begin{minipage}[ct]{0.9\linewidth}
    \textsc{SF Hacks}\hfill May 2022 to Present\\
    \emph{Graduate Mentor}, May 2024 to Present\\
    \emph{Treasurer}, May 2022 to May 2024
\end{minipage}

\vspace{\contentblockspacing}

\begin{minipage}[ct]{0.9\linewidth}
    \textsc{CS}\{\textsc{Research}\}\textsc{ Club}\hfill Aug 2023 to Present\\
    \emph{President/Founder}, Aug 2023 to Present
\end{minipage}

\vspace{\contentblockspacing}

\begin{minipage}[ct]{0.9\linewidth}
    \textsc{Artificial Intelligence Club}\hfill Aug 2023 to Present\\
    \emph{Treasurer}, Aug 2023 to Present
\end{minipage}

\vspace{\contentblockspacing}

\begin{minipage}[ct]{0.9\linewidth}
    \textsc{Korean Student Association}\hfill Aug 2024 to Present\\
    \emph{President}, Aug 2024 to Present
\end{minipage}

\vspace{\sectionspacing}

%%%%%%%%%%%%%%%%%%%%%%%%%%%%%%%%%%%%%%%%%%%%%%%%%%%%%
% Professional Memberships
%%%%%%%%%%%%%%%%%%%%%%%%%%%%%%%%%%%%%%%%%%%%%%%%%%%%%

\section{Professional Memberships}
\vspace{\sectiontocontentspacing}

\begin{minipage}[ct]{0.9\linewidth}
    \textsc{Society for Industrial and Applied Mathematics} (SIAM)
\end{minipage}

\vspace{0.5mm}

\begin{minipage}[ct]{0.9\linewidth}
    \textsc{Association for Computing Machinery} (ACM)
\end{minipage}

\vspace{0.5mm}

\begin{minipage}[ct]{0.9\linewidth}
    \textsc{National Association of School Psychologists} (NASP)
\end{minipage}

\vspace{0.5mm}

\begin{minipage}[ct]{0.9\linewidth}
    \textsc{Institute of Electrical and Electronics Engineers} (IEEE)
\end{minipage}

\vspace{\sectionspacing}

%%%%%%%%%%%%%%%%%%%%%%%%%%%%%%%%%%%%%%%%%%%%%%%%%%%%%
% PROFESSIONAL EXPERIENCE
%%%%%%%%%%%%%%%%%%%%%%%%%%%%%%%%%%%%%%%%%%%%%%%%%%%%%
\section{Professional Experience}

\vspace{\sectiontocontentspacing}

\begin{minipage}[ct]{\contentwidth}
    \textbf{Financial Center Manager}, AVP\hfill 2017 to 2018\\
    {\textsc{Bank of America} - Belmont, CA}
\end{minipage}

\vspace{\contentblockspacing}

\begin{minipage}[ct]{0.9\linewidth}
    \textbf{Founder/CEO}\hfill 2005 to 2017\\
    {\textsc{Kindred Enterprises Incorporated} - San Francisco, CA}
\end{minipage}

\vspace{\contentblockspacing}

\begin{minipage}[ct]{0.9\linewidth}
    \textbf{Landing Support Specialist}, Corporal\hfill 1995 to 2001\\
    {\textsc{United States Marine Corps Reserve} - San Jose, CA}
\end{minipage}

\vspace{\sectionspacing}

%%%%%%%%%%%%%%%%%%%%%%%%%%%%%%%%%%%%%%%%%%%%%%%%%%%%%
% References
%%%%%%%%%%%%%%%%%%%%%%%%%%%%%%%%%%%%%%%%%%%%%%%%%%%%%

\section{References}

\vspace{\sectiontocontentspacing}

\begin{minipage}[ct]{\contentwidth}
    \begin{multicols}{2}
        \textbf{Hui Yang}\\
        \emph{Associate Professor}\\
        {Department of Computer Science}\\
        \textsc{San Francisco State University}\\
        \href{tel:4153382221}{415.338.2221}, \href{mailto:huiyang@sfsu.edu}{huiyang@sfsu.edu}\\

        \vspace{-1mm}

        \textbf{Arno Puder}\\
        \emph{Professor and Department Chair}\\
        {Department of Computer Science}\\
        \textsc{San Francisco State University}\\
        \href{tel:4153382853}{415.338.2853}, \href{mailto:arno@sfsu.edu}{arno@sfsu.edu}\\

        \vspace{-1mm}

        \textbf{Jessica Fielder}\\
        \emph{Supplemental Instruction Program Director}\\
        {Center for Science and Mathematics Education}\\
        \textsc{San Francisco State University}\\
        \href{tel:4154050540}{415.405.0540}, \href{mailto:jfielder@sfsu.edu}{jfielder@sfsu.edu}\\

        \columnbreak

        \textbf{Anagha Kulkarni}\\
        \emph{Professor and Associate Department Chair}\\
        {Department of Computer Science}\\
        \textsc{San Francisco State University}\\
        \href{tel:4153382539}{415.338.2539}, \href{mailto:ak@sfsu.edu}{ak@sfsu.edu}\\

        \vspace{-1mm}

        \textbf{Shasta Ihorn}\\
        \emph{Associate Professor}\\
        {Department of Psychology}\\
        \textsc{San Francisco State University}\\
        \href{tel:4153383218}{415.338.3218}, \href{mailto:sihorn@sfsu.edu}{sihorn@sfsu.edu}\\
    \end{multicols}
\end{minipage}

%%%%%%%%%%%%%%%%%%%%%%%%%%%%%%%%%%%%%%%%%%%%%%%%%%%%%
% SKILLS & COMPETENCIES
%%%%%%%%%%%%%%%%%%%%%%%%%%%%%%%%%%%%%%%%%%%%%%%%%%%%%

% \section{Skills \& Competencies}

% \vspace{3mm}

% \begin{minipage}[ct]{0.9\linewidth}
%     \textsc{Languages: } Python, Java, C/C++, JavaScript, HTML, CSS, MySQL, Matlab, R

%     % \vspace{1.5mm}

%     % \textsc{Operating Systems: } Windows, MacOS, Linux (Ubuntu, Debian)

%     \vspace{1.5mm}

%     \textsc{Web Frameworks/Environments: } Node.js, Express.js, React.js, Handlebars.js

%     \vspace{1.5mm}

%     \textsc{Libraries: } PyTorch, Pandas, NumPy, scikit-learn, Matplotlib,
%     BeautifulSoup, Plotly, Dash

%     \vspace{1.5mm}

%     \textsc{Deployment/Cloud Computing:} AWS (EC$2$, S$3$, and Route$53$),
%     Google Cloud (Compute Engine, Storage, Domains), Google Analytics, Oracle Cloud Infrastructure, NGINX

%     \vspace{1.5mm}

%     \textsc{Development Platforms: } Docker, Git, Jupyter, Conda

%     % \vspace{1.5mm}

%     % \textsc{Development Environments: } VSCode, IntelliJ IDEA, Visual Studio,
%     % PyCharm

%     \vspace{1.5mm}

%     \textsc{Other: } \LaTeX, Adobe Creative Suite (XD, Illustrator, InDesign, Photoshop)
% \end{minipage}

}

\begin{tikzpicture}[overlay,remember picture]
    \node (address) at ($ (current page.south east) - (1.5,-3.6)$)
    {\href{tel:\phonenumber}{\includegraphics[scale=0.08]{phone.png}}};
    \node (address) at ($ (current page.south east) - (1.5,-2.9)$)
    {\href{mailto:\email}{\includegraphics[scale=0.08]{env.png}}};
    \node (address) at ($ (current page.south east) - (1.5,-2.2)$)
    {\href{\linkedinurl}{\includegraphics[scale=0.08]{li.png}}};
    \node (address) at ($ (current page.south east) - (1.5,-1.5)$)
    {\href{\githuburl}{\includegraphics[scale=0.08]{gh.png}}};
    \draw [fill=white,draw=none]
    ($ (current page.south east) - (1.9cm,-0.5cm) $)
    rectangle
    ($ (current page.south east) - (6.7cm,-1.5cm) $);
    \node [draw=none] at ($ (current page.south east) - (4.3,-1.01)$) {\fontsize{23}{23}\selectfont
        \textbf{Mark S. Kim}};
\end{tikzpicture}

\end{document}

% \hypersetup{
% pdftitle={Resume},
% pdfsubject={Resume},
% pdfauthor={Mark Kim}
% pdfkeywords={YA.NET (A\#/A sharp), A-0 System, A+ (A plus), A++, ABAP, ABC, ABC
% ALGOL, ACC, Accent (Rational Synergy), Ace DASL (Distributed Application
% Specification Language), Action!, ActionScript, Actor, Ada, Adenine (Haystack),
% AdvPL, Agda, Agilent VEE (Keysight VEE), Agora, AIMMS, Aldor, Alef, ALF, ALGOL
% 58, ALGOL 60, ALGOL 68, ALGOL W, Alice (Alice ML), Alma-0, AmbientTalk, Amiga E,
% AMOS (AMOS BASIC), AMPL, Analitik, AngelScript, Apache Pig latin, Apex
% (Salesforce.com, Inc), APL, App Inventor for Android's visual block language
% (MIT App Inventor), AppleScript, APT, Arc, ARexx, Argus, Assembly language
% (ASM), AutoHotkey, AutoIt, AutoLISP / Visual LISP, Averest, AWK, Axum, Babbage,
% Ballerina, Bash, BASIC, Batch file (Windows/MS-DOS), bc (basic calculator),
% BCPL, BeanShell, Bertrand, BETA, BLISS, Blockly, BlooP, Boo, Boomerang, Bosque,
% C – ISO/IEC 9899, C-- (C minus minus), C++ (C plus plus) – ISO/IEC 14882, C*,
% C\# (C sharp) – ISO/IEC 23270, C/AL, Caché ObjectScript, C Shell (csh), Caml,
% Carbon, Cayenne (Lennart Augustsson), CDuce, Cecil, CESIL (Computer Education in
% Schools Instruction Language), Céu, Ceylon, CFEngine, Cg (High-Level
% Shader/Shading Language [HLSL]), Ch, Chapel (Cascade High Productivity
% Language), Charm, CHILL, CHIP-8, ChucK, Cilk (also Cilk++ and Cilk plus),
% Control Language, Claire, Clarion, Clean, Clipper, CLIPS, CLIST, Clojure, CLU,
% CMS-2, COBOL – ISO/IEC 1989, CobolScript – COBOL Scripting language, Cobra,
% CoffeeScript, ColdFusion, COMAL, COMIT, Common Intermediate Language (CIL),
% Common Lisp (also known as CL), COMPASS, Component Pascal, Constraint Handling
% Rules (CHR), COMTRAN, Cool, Coq, Coral 66, CorVision, COWSEL, CPL, Cryptol,
% Crystal, Csound, Cuneiform, Curl, Curry, Cybil, Cyclone, Cypher Query Language,
% Cython, CEEMAC, Dart, Darwin, DataFlex, Datalog, DATATRIEVE, dBase, dc, DCL
% (DIGITAL Command Language), Delphi, DinkC, DIBOL, Dog, Draco, DRAKON, Dylan,
% DYNAMO, DAX (Data Analysis Expressions), E, Ease, Easy PL/I, EASYTRIEVE PLUS,
% eC, ECMAScript, Edinburgh IMP, EGL, Eiffel, ELAN, Elixir, Elm, Emacs Lisp,
% Emerald, Epigram, EPL (Easy Programming Language), Erlang, es, Escher, ESPOL,
% Esterel, Etoys, Euclid, Euler, Euphoria, EusLisp Robot Programming Language, CMS
% EXEC (EXEC), EXEC 2, Executable UML, Ezhil, F\# (F sharp), F*, Factor, Fantom,
% FAUST, FFP, fish, Fjölnir, FL, FlagShip, Flavors, Flex, Flix, FlooP, FLOW-MATIC
% (B0), FOCAL (Formulating On-Line Calculations in Algebraic Language/FOrmula
% CALculator), FOCUS, FOIL, FORMAC (FORMula MAnipulation Compiler), @Formula,
% Forth, Fortran – ISO/IEC 1539, Fortress, FP, FoxBase/FoxPro, Franz Lisp,
% Futhark, F-Script, Game Maker Language, GameMonkey Script, GAMS (General
% Algebraic Modeling System), GAP, G-code, GDScript (Godot), Genie, GDL (Geometric
% Description Language), GEORGE, GLSL (OpenGL Shading Language), GNU E, GNU Guile
% (GNU Ubiquitous Intelligent Language for Extensions), Go, Go!, GOAL (Game
% Oriented Assembly Lisp), Gödel, Golo, GOM (Good Old Mad), Google Apps Script,
% Gosu, GOTRAN (IBM 1620), GPSS (General Purpose Simulation System), GraphTalk
% (Computer Sciences Corporation), GRASS, Grasshopper, Groovy (Apache Groovy),
% Hack, HAGGIS, HAL/S, Halide (programming language), Hamilton C shell, Harbour,
% Hartmann pipelines, Haskell, Haxe, Hermes, High Level Assembly (HLA), HLSL,
% Hollywood, HolyC (TempleOS), Hop, Hopscotch, Hope, Hume, HyperTalk, Hy, Io,
% Icon, IBM Basic assembly language, IBM HAScript, IBM Informix-4GL, IBM RPG, IDL,
% Idris, Inform, ISLISP, J\# (J sharp), J++ (J plus plus), JADE, Jai, JAL, Janus
% (concurrent constraint programming language), Janus (time-reversible computing
% programming language), JASS, Java, JavaFX Script, JavaScript, Jess, JCL, JEAN,
% Join Java, JOSS, Joule, JOVIAL, Joy, JScript, JScript .NET, Julia, Jython,
% Kaleidoscope, Karel, KEE, Kixtart, Klerer-May System, KIF (Knowledge Interchange
% Format), Kojo, Kotlin, KRC, KRL, KRL (KUKA Robot Language), KRYPTON, KornShell
% (ksh), Kodu, Kv (Kivy), LabVIEW, Ladder, LANSA, Lasso, Lava, LC-3, Lean,
% Legoscript, LIL, LilyPond, Limbo, Limnor, LINC, Lingo, LINQ, LIS, LISA, Language
% H, Lisp – ISO/IEC 13816, Lite-C, Lithe, Little b, LLL, Logo, Logtalk,
% LotusScript, LPC, LSE, LSL, LiveCode, LiveScript, Lua, Lucid, Lustre, LYaPAS,
% Lynx, M Formula language, M2001, M4, M\#, Machine code, MAD (Michigan Algorithm
% Decoder), MAD/I, Magik, Magma, Máni, Maple, MAPPER (now part of BIS), MARK-IV
% (now VISION:BUILDER), Mary, MATLAB, MASM Microsoft Assembly x86, MATH-MATIC,
% Maude system, Maxima (see also Macsyma), Max (Max Msp – Graphical Programming
% Environment), MaxScript internal language 3D Studio Max, Maya (MEL), MDL,
% Mercury, Mesa, MHEG-5 (Interactive TV programming language), Microcode,
% Microsoft Power Fx, MIIS, Milk (programming language), MIMIC, Mirah, Miranda,
% MIVA Script, ML, Model 204, Modelica, Modula, Modula-2, Modula-3, Mohol, MOO,
% Mortran, Mouse, MPD, MSL, MUMPS, MuPAD, Mutan, Mystic Programming Language
% (MPL), NASM, Napier88, Neko, Nemerle, NESL, Net.Data, NetLogo, NetRexx, NewLISP,
% NEWP, Newspeak, NewtonScript, Nial, Nickle (NITIN), Nim, Nix (Systems
% configuration language), NPL, Not eXactly C (NXC), Not Quite C (NQC), NSIS, Nu,
% NWScript, NXT-G, o:XML, Oak, Oberon, OBJ2, Object Lisp, ObjectLOGO, Object REXX,
% Object Pascal, Objective-C, Obliq, OCaml, occam, occam-pi, Octave, OmniMark,
% Opa, Opal, Open Programming Language (OPL), OpenCL, OpenEdge Advanced Business
% Language (ABL), OpenVera, OpenQASM, OPS5, OptimJ, Orc, ORCA/Modula-2, Oriel,
% Orwell, Oxygene, Oz, P, P4, ParaSail, PARI/GP, Pascal – ISO 7185, Pascal Script,
% PCASTL, PCF, PEARL, PeopleCode, Perl, PDL, Pharo, PHP, Pico, Picolisp, Pict,
% Pike, PILOT, Pipelines, Pizza, PL-11, PL/0, PL/B, PL/C, PL/I – ISO 6160, PL/M,
% PL/P, PL/S, PL/SQL, PL360, PLANC, Plankalkül, Planner, PLEX, PLEXIL, Plus,
% POP-11, POP-2, PostScript, PortablE, POV-Ray SDL, Powerhouse, PowerBuilder – 4GL
% GUI application generator from Sybase, PowerShell, PPL, Processing,
% Processing.js, Prograph, Project Verona, Prolog, PROMAL, Promela, PROSE modeling
% language, PROTEL, ProvideX, Pro*C, Pure, Pure Data, PureScript, PWCT, Python, Q
% (programming language from Kx Systems), Q\# (Microsoft programming language),
% Qalb, Quantum Computation Language, QtScript, QuakeC, QPL, .QL, R, R++, Racket,
% Raku, RAPID, Rapira, Ratfiv, Ratfor, rc, Reason, REBOL, Red, Redcode, REFAL,
% REXX, Ring, ROOP, RPG, RPL, RSL, RTL/2, Ruby, Rust, S, S2, S3, S-Lang, S-PLUS,
% SA-C, SabreTalk, SAIL, SAKO, SAS, SASL, Sather, Sawzall, Scala, Scheme, Scilab,
% Scratch, Script.NET, Sed, Seed7, Self, SenseTalk, SequenceL, Serpent, SETL,
% Short Code, SIMPOL, SIGNAL, SiMPLE, SIMSCRIPT, Simula, Simulink, SISAL, SKILL,
% SLIP, SMALL, Smalltalk, SML, Strongtalk, Snap!, SNOBOL (SPITBOL), Snowball, SOL,
% Solidity, SOPHAEROS, Source, SPARK, Speakeasy, Speedcode, SPIN, SP/k, SPS, SQL,
% SQR, Squeak, Squirrel, SR, S/SL, Starlogo, Strand, Stata, Stateflow, Subtext,
% SBL, SuperCollider, Superplan, SuperTalk, Swift (Apple programming language),
% Swift (parallel scripting language), SYMPL, TACL, TACPOL, TADS (Text Adventure
% Development System), TAL, Tcl, Tea, TECO (Text Editor and Corrector), TELCOMP,
% TeX, TEX (Text Executive Programming Language), TIE, TMG (TransMo Griffer),
% compiler-compiler, Tom, Toi, Topspeed (Clarion), TPU (Text Processing Utility),
% Trac, TTM, T-SQL (Transact-SQL), Transcript (LiveCode), TTCN (Tree and Tabular
% Combined Notation), Turing, TUTOR (PLATO Author Language), TXL, TypeScript,
% Tynker, Ubercode, UCSD Pascal, Umple, Unicon, Uniface, UNITY, UnrealScript,
% Vala, Vim script, Viper (Ethereum/Ether (ETH)), Visual DataFlex, Visual
% DialogScript, Visual FoxPro, Visual J++ (Visual J plus plus), Visual LISP,
% Visual Objects, Visual Prolog, WATFIV, WATFOR (WATerloo FORtran IV),
% WebAssembly, WebDNA, Whiley, Winbatch, Wolfram Language, Wyvern, X++ (X plus
% plus/Microsoft Dynamics AX), X10, xBase++ (xBase plus plus), XBL, XC (targets
% XMOS architecture), xHarbour, XL, Xojo, XOTcl, Xod, XPL, XPL0, XQuery, XSB,
% XSharp (X\#), XSLT, Xtend, Yorick, YQL, Yoix, Z notation, Z shell, Zebra, ZPL,
% ZPL2, Zeno, ZetaLisp, Zig, Zonnon, ZOPL, ZPL, Z++}
% }

% \newcommand{\altura}{.45cm}
% \begin{tikzpicture}[remember picture, overlay, x=1cm, y=-\altura, node distance=0,outer sep=0,inner sep=0, opacity=0]
%     \tikzset{xshift=-1cm,yshift=-0.2cm}
%     \tikzstyle{nome}=[draw, rectangle,anchor=west, minimum height=\altura,
%     minimum width=25cm,fill=yellow!30,text width=25cm, scale=0.75]
%     \node[nome] (p0) {\parbox{25cm}{A.NET (A\#/A sharp), A-0 System, A+ (A plus), A++, ABAP, ABC, ABC ALGOL, ACC, Accent (Rational Synergy), MATLAB, Net.Data, XSB, XSharp (X\#), }};
%     \node[nome] (p1) [below = of p0] {\parbox{25cm}{Ace DASL (Distributed Application Specification Language), Action!, ActionScript, Actor, Ada, Adenine (Haystack), AdvPL, Agda, XSLT, Xtend, }};
%     \node[nome] (p2) [below = of p1] {\parbox{25cm}{Agilent VEE (Keysight VEE), Agora, AIMMS, Aldor, Alef, ALF, ALGOL 58, ALGOL 60, ALGOL 68, ALGOL W, Alice (Alice ML), Alma-0, AmbientTalk, }};
%     \node[nome] (p3) [below = of p2] {\parbox{25cm}{Amiga E, AMOS (AMOS BASIC), AMPL, Analitik, AngelScript, Apache Pig latin, Apex (Salesforce.com, Inc), APL, X++ (X plus plus/Microsoft Dynamics AX), }};
%     \node[nome] (p4) [below = of p3] {\parbox{25cm}{App Inventor for Android's visual block language (MIT App Inventor), AppleScript, APT, Arc, ARexx, Argus, Assembly language (ASM), AutoHotkey, }};
%     \node[nome] (p5) [below = of p4] {\parbox{25cm}{AutoIt, AutoLISP / Visual LISP, Averest, AWK, Axum, Babbage, Ballerina, Bash, BASIC, Batch file (Windows/MS-DOS), bc (basic calculator), BCPL, }};
%     \node[nome] (p6) [below = of p5] {\parbox{25cm}{BeanShell, Bertrand, BETA, BLISS, Blockly, BlooP, Boo, Boomerang, Bosque, C – ISO/IEC 9899, C-- (C minus minus), Xojo, XOTcl, Xod, XPL, }};
%     \node[nome] (p7) [below = of p6] {\parbox{25cm}{C++ (C plus plus) – ISO/IEC 14882, C*, C\# (C sharp) – ISO/IEC 23270, C/AL, Caché ObjectScript, C Shell (csh), Caml, Carbon, XPL0, XQuery, }};
%     \node[nome] (p8) [below = of p7] {\parbox{25cm}{Cayenne (Lennart Augustsson), CDuce, Cecil, CESIL (Computer Education in Schools Instruction Language), Céu, Ceylon, CFEngine, Yorick, YQL, }};
%     \node[nome] (p9) [below = of p8] {\parbox{25cm}{Cg (High-Level Shader/Shading Language [HLSL]), Ch, Chapel (Cascade High Productivity Language), Charm, CHILL, CHIP-8, ChucK, Yoix, Z notation, }};
%     \node[nome] (p10) [below = of p9] {\parbox{25cm}{Cilk (also Cilk++ and Cilk plus), Control Language, Claire, Clarion, Clean, Clipper, CLIPS, CLIST, Clojure, CLU, CMS-2, COBOL – ISO/IEC 1989, }};
%     \node[nome] (p11) [below = of p10] {\parbox{25cm}{CobolScript – COBOL Scripting language, Cobra, CoffeeScript, ColdFusion, COMAL, COMIT, Common Intermediate Language (CIL), }};
%     \node[nome] (p12) [below = of p11] {\parbox{25cm}{Common Lisp (also known as CL), COMPASS, Component Pascal, Constraint Handling Rules (CHR), COMTRAN, Cool, Coq, Coral 66, CorVision, }};
%     \node[nome] (p13) [below = of p12] {\parbox{25cm}{CPL, Cryptol, Crystal, Csound, Cuneiform, Curl, Curry, Cybil, Cyclone, Cypher Query Language, Cython, CEEMAC, Dart, Darwin, DataFlex, Datalog, }};
%     \node[nome] (p14) [below = of p13] {\parbox{25cm}{DATATRIEVE, dBase, dc, DCL (DIGITAL Command Language), Delphi, DinkC, DIBOL, Dog, Draco, DRAKON, Dylan, DYNAMO, COWSEL, }};
%     \node[nome] (p15) [below = of p14] {\parbox{25cm}{DAX (Data Analysis Expressions), E, Ease, Easy PL/I, EASYTRIEVE PLUS, eC, ECMAScript, Edinburgh IMP, EGL, Eiffel, ELAN, Elixir, Elm, }};
%     \node[nome] (p16) [below = of p15] {\parbox{25cm}{Emacs Lisp, Emerald, Epigram, EPL (Easy Programming Language), Erlang, es, Escher, ESPOL, Esterel, Etoys, Euclid, Euler, Euphoria, Z shell, Zebra, ZPL, }};
%     \node[nome] (p17) [below = of p16] {\parbox{25cm}{EusLisp Robot Programming Language, CMS EXEC (EXEC), EXEC 2, Executable UML, Ezhil, F\# (F sharp), F*, Factor, Fantom, FAUST, FFP, fish, }};
%     \node[nome] (p18) [below = of p17] {\parbox{25cm}{Fjölnir, FL, FlagShip, Flavors, Flex, Flix, FlooP, FLOW-MATIC (B0), X10, xBase++ (xBase plus plus), XBL, XC (targets XMOS architecture), xHarbour, XL, }};
%     \node[nome] (p19) [below = of p18] {\parbox{25cm}{FOCAL (Formulating On-Line Calculations in Algebraic Language/FOrmula CALculator), FOCUS, FOIL, FORMAC (FORMula MAnipulation Compiler), }};
%     \node[nome] (p20) [below = of p19] {\parbox{25cm}{@Formula, Forth, Fortran – ISO/IEC 1539, Fortress, FP, FoxBase/FoxPro, Franz Lisp, Futhark, F-Script, Game Maker Language, GameMonkey Script, }};
%     \node[nome] (p21) [below = of p20] {\parbox{25cm}{GAMS (General Algebraic Modeling System), GAP, G-code, GDScript (Godot), Genie, GDL (Geometric Description Language), GEORGE, Zonnon, ZOPL, }};
%     \node[nome] (p22) [below = of p21] {\parbox{25cm}{GLSL (OpenGL Shading Language), GNU E, GNU Guile (GNU Ubiquitous Intelligent Language for Extensions), Go, Go!, ZPL2, Zeno, ZetaLisp, Zig, }};
%     \node[nome] (p23) [below = of p22] {\parbox{25cm}{GOAL (Game Oriented Assembly Lisp), Gödel, Golo, GOM (Good Old Mad), Google Apps Script, Gosu, GOTRAN (IBM 1620), ZPL, Z++}};
%     \node[nome] (p24) [below = of p23] {\parbox{25cm}{GPSS (General Purpose Simulation System), GraphTalk (Computer Sciences Corporation), GRASS, Grasshopper, Groovy (Apache Groovy), Hack, HAGGIS, }};
%     \node[nome] (p25) [below = of p24] {\parbox{25cm}{HAL/S, Halide (programming language), Hamilton C shell, Harbour, Hartmann pipelines, Haskell, Haxe, Hermes, High Level Assembly (HLA), HLSL, }};
%     \node[nome] (p26) [below = of p25] {\parbox{25cm}{Hollywood, HolyC (TempleOS), Hop, Hopscotch, Hope, Hume, HyperTalk, Hy, Io, Icon, IBM Basic assembly language, IBM HAScript, IBM Informix-4GL, }};
%     \node[nome] (p27) [below = of p26] {\parbox{25cm}{IBM RPG, IDL, Idris, Inform, ISLISP, J\# (J sharp), J++ (J plus plus), JADE, Jai, JAL, Janus (concurrent constraint programming language), }};
%     \node[nome] (p28) [below = of p27] {\parbox{25cm}{Janus (time-reversible computing programming language), JASS, Java, JavaFX Script, JavaScript, Jess, JCL, JEAN, Join Java, JOSS, Joule, JOVIAL, }};
%     \node[nome] (p29) [below = of p28] {\parbox{25cm}{Joy, JScript, JScript .NET, Julia, Jython, Kaleidoscope, Karel, KEE, Kixtart, Klerer-May System, KIF (Knowledge Interchange Format), Kojo, }};
%     \node[nome] (p30) [below = of p29] {\parbox{25cm}{Kotlin, KRC, KRL, KRL (KUKA Robot Language), KRYPTON, KornShell (ksh), Kodu, Kv (Kivy), LabVIEW, Ladder, LANSA, Lasso, Lava, LC-3, Lean, }};
%     \node[nome] (p31) [below = of p30] {\parbox{25cm}{Legoscript, LIL, LilyPond, Limbo, Limnor, LINC, Lingo, LINQ, LIS, LISA, Language H, Lisp – ISO/IEC 13816, Lite-C, Lithe, Little b, LLL, Logo, }};
%     \node[nome] (p32) [below = of p31] {\parbox{25cm}{Logtalk, LotusScript, LPC, LSE, LSL, LiveCode, LiveScript, Lua, Lucid, Lustre, LYaPAS, Lynx, M Formula language, M2001, M4, M\#, Machine code, }};
%     \node[nome] (p33) [below = of p32] {\parbox{25cm}{MAD (Michigan Algorithm Decoder), MAD/I, Magik, Magma, Máni, Maple, MAPPER (now part of BIS), MARK-IV (now VISION:BUILDER), Mary, }};
%     \node[nome] (p34) [below = of p33] {\parbox{25cm}{MASM Microsoft Assembly x86, MATH-MATIC, Maude system, Maxima (see also Macsyma), Max (Max Msp – Graphical Programming Environment), }};
%     \node[nome] (p35) [below = of p34] {\parbox{25cm}{MaxScript internal language 3D Studio Max, Maya (MEL), MDL, Mercury, Mesa, MHEG-5 (Interactive TV programming language), Microcode, }};
%     \node[nome] (p36) [below = of p35] {\parbox{25cm}{Microsoft Power Fx, MIIS, Milk (programming language), MIMIC, Mirah, Miranda, MIVA Script, ML, Model 204, Modelica, Modula, Modula-2, Modula-3, }};
%     \node[nome] (p37) [below = of p36] {\parbox{25cm}{Mohol, MOO, Mortran, Mouse, MPD, MSL, MUMPS, MuPAD, Mutan, Mystic Programming Language (MPL), NASM, Napier88, Neko, Nemerle, NESL, }};
%     \node[nome] (p38) [below = of p37] {\parbox{25cm}{NetLogo, NetRexx, NewLISP, NEWP, Newspeak, NewtonScript, Nial, Nickle (NITIN), Nim, Nix (Systems configuration language), NPL, }};
%     \node[nome] (p39) [below = of p38] {\parbox{25cm}{Not eXactly C (NXC), Not Quite C (NQC), NSIS, Nu, NWScript, NXT-G, o:XML, Oak, Oberon, OBJ2, Object Lisp, ObjectLOGO, Object REXX, }};
%     \node[nome] (p40) [below = of p39] {\parbox{25cm}{Object Pascal, Objective-C, Obliq, OCaml, occam, occam-$\pi$, Octave, OmniMark, Opa, Opal, Open Programming Language (OPL), OpenCL, }};
%     \node[nome] (p41) [below = of p40] {\parbox{25cm}{OpenEdge Advanced Business Language (ABL), OpenVera, OpenQASM, OPS5, OptimJ, Orc, ORCA/Modula-2, Oriel, Orwell, Oxygene, Oz, P, P4, ParaSail, }};
%     \node[nome] (p42) [below = of p41] {\parbox{25cm}{PARI/GP, Pascal – ISO 7185, Pascal Script, PCASTL, PCF, PEARL, PeopleCode, Perl, PDL, Pharo, PHP, Pico, Picolisp, Pict, Pike, PILOT, Pipelines, }};
%     \node[nome] (p43) [below = of p42] {\parbox{25cm}{Pizza, PL-11, PL/0, PL/B, PL/C, PL/I – ISO 6160, PL/M, PL/P, PL/S, PL/SQL, PL360, PLANC, Plankalkül, Planner, PLEX, PLEXIL, Plus, POP-11, }};
%     \node[nome] (p44) [below = of p43] {\parbox{25cm}{POP-2, PostScript, PortablE, POV-Ray SDL, Powerhouse, PowerBuilder – 4GL GUI application generator from Sybase, PowerShell, PPL, Processing, }};
%     \node[nome] (p45) [below = of p44] {\parbox{25cm}{Processing.js, Prograph, Project Verona, Prolog, PROMAL, Promela, PROSE modeling language, PROTEL, ProvideX, Pro*C, Pure, Pure Data, }};
%     \node[nome] (p46) [below = of p45] {\parbox{25cm}{PureScript, PWCT, Python, Q (programming language from Kx Systems), Q\# (Microsoft programming language), Qalb, Quantum Computation Language, }};
%     \node[nome] (p47) [below = of p46] {\parbox{25cm}{QtScript, QuakeC, QPL, .QL, R, R++, Racket, Raku, RAPID, Rapira, Ratfiv, Ratfor, rc, Reason, REBOL, Red, Redcode, REFAL, REXX, Ring, ROOP, RPG, }};
%     \node[nome] (p48) [below = of p47] {\parbox{25cm}{RPL, RSL, RTL/2, Ruby, Rust, S, S2, S3, S-Lang, S-PLUS, SA-C, SabreTalk, SAIL, SAKO, SAS, SASL, Sather, Sawzall, Scala, Scheme, Scilab, }};
%     \node[nome] (p49) [below = of p48] {\parbox{25cm}{Scratch, Script.NET, Sed, Seed7, Self, SenseTalk, SequenceL, Serpent, SETL, Short Code, SIMPOL, SIGNAL, SiMPLE, SIMSCRIPT, Simula, Simulink, }};
%     \node[nome] (p50) [below = of p49] {\parbox{25cm}{SISAL, SKILL, SLIP, SMALL, Smalltalk, SML, Strongtalk, Snap!, SNOBOL (SPITBOL), Snowball, SOL, Solidity, SOPHAEROS, Source, SPARK, Speakeasy, }};
%     \node[nome] (p51) [below = of p50] {\parbox{25cm}{Speedcode, SPIN, SP/k, SPS, SQL, SQR, Squeak, Squirrel, SR, S/SL, Starlogo, Strand, Stata, Stateflow, Subtext, SBL, SuperCollider, Superplan, }};
%     \node[nome] (p52) [below = of p51] {\parbox{25cm}{SuperTalk, Swift (Apple programming language), Swift (parallel scripting language), SYMPL, TACL, TACPOL, }};
%     \node[nome] (p53) [below = of p52] {\parbox{25cm}{TADS (Text Adventure Development System), TAL, Tcl, Tea, TECO (Text Editor and Corrector), TELCOMP, TeX, }};
%     \node[nome] (p54) [below = of p53] {\parbox{25cm}{TEX (Text Executive Programming Language), TIE, TMG (TransMo Griffer), compiler-compiler, Tom, Toi, Topspeed (Clarion), }};
%     \node[nome] (p55) [below = of p54] {\parbox{25cm}{TPU (Text Processing Utility), Trac, TTM, T-SQL (Transact-SQL), Transcript (LiveCode), TTCN (Tree and Tabular Combined Notation), Turing, }};
%     \node[nome] (p56) [below = of p55] {\parbox{25cm}{TUTOR (PLATO Author Language), TXL, TypeScript, Tynker, Ubercode, UCSD Pascal, Umple, Unicon, Uniface, UNITY, UnrealScript, Vala, Vim script, }};
%     \node[nome] (p57) [below = of p56] {\parbox{25cm}{Viper (Ethereum/Ether (ETH)), Visual DataFlex, Visual DialogScript, Visual FoxPro, Visual J++ (Visual J plus plus), Visual LISP, }};
%     \node[nome] (p58) [below = of p57] {\parbox{25cm}{Visual Objects, Visual Prolog, WATFIV, WATFOR (WATerloo FORtran IV), WebAssembly, WebDNA, Whiley, Winbatch, Wolfram Language, Wyvern, }};
%     \node[nome] (p59) [below = of p58] {\parbox{25cm}{}};
%     \node[nome] (p60) [below = of p59] {\parbox{25cm}{}};
%     \node[nome] (p61) [below = of p60] {\parbox{25cm}{}};
%     \node[nome] (p62) [below = of p61] {\parbox{25cm}{}};
%     \node[nome] (p63) [below = of p62] {\parbox{25cm}{}};
%     \node[nome] (p64) [below = of p63] {\parbox{25cm}{}};
%     \node[nome] (p65) [below = of p64] {\parbox{25cm}{}};
%     \node[nome] (p66) [below = of p65] {\parbox{25cm}{}};
%     \node[nome] (p67) [below = of p66] {\parbox{25cm}{}};
%     \node[nome] (p68) [below = of p67] {\parbox{25cm}{}};
%     \node[nome] (p69) [below = of p68] {\parbox{25cm}{}};
%     \node[nome] (p70) [below = of p69] {\parbox{25cm}{}};
%     \node[nome] (p71) [below = of p70] {\parbox{25cm}{}};
%     \node[nome] (p72) [below = of p71] {\parbox{25cm}{}};
% \end{tikzpicture}