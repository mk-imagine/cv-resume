\documentclass[12pt]{article}
\usepackage[letterpaper]{geometry}
\usepackage{setspace}
\geometry{top=.9in, bottom=.9in, left=.9in, right=.9in}
\doublespacing

\makeatletter
\def\@maketitle{%
  \newpage
  \null
  \vskip 2em%
  \begin{center}%
    {\large
        \begin{tabular}[t]{c}%
        \@author
        \end{tabular}\par}%
  \let \footnote \thanks
    {\LARGE \@title \par}%
    %{\large \@date}%
  \end{center}%
  \par
  \vskip 1.5em}
\makeatother

\title{Diversity Statement}
\author{Mark Kim}

\begin{document}
\maketitle

When I re-entered college I never expected to have an interest in serving a
college or university as a member of the faculty.  In fact, my return to school was
predicated entirely on a business idea and personal interest based in fueling
people's career decisions from latent passions that may be found through data
mining.  However, my perspective shifted when I joined a student-led supplemental instruction course in introductory programming, which
sparked an interest in teaching.  Studying at San Francisco State University, one of the most ethnically diverse universities in the U.S.,
has enriched my understanding of different cultures, perspectives, and experiences. This vibrant, inclusive environment inspired me to
contribute to the academic community, much like the faculty members who mentored me.

The supplemental instruction program that I so fondly participated in as a
pupil unveiled the possibility that I, even as an undergraduate student, could
also lead and facilitate such a course.  Before this, I assumed that my
knowledge and past endeavors in the field of Computer Science and Mathematics were
insufficient to be of value to other students.  Still uncertain of my abilities,
I found a position grading math homework, which not only solidified my understanding but also helped me recognize the impact of supporting
others' learning.

During my stint as a grader, and in response to COVID-19 pandemic lockdowns, I
was provided the chance to assist a professor with a research project on remote learning pedagogy.  Both fascinating and eye-opening
to me, this study provided me with insight into challenges of providing a rigorous education amid financial and racial inequities.  This
experience made me reflect deeply on how I would teach students in Math or Computer Science if given the chance.

Shortly after completing my brief encounter with Mathematics education research,
I was awarded the opportunity to become a supplemental instruction facilitator
in Calculus and Computer Science.  Armed with my prior experiences, the prospect of applying my newly
gained knowledge thrilled me.  I was determined to apply what I learned and attempt to
deliver an experience that not only inspired students, but also significantly
improved their learning outcomes.  My time in this program was deeply fulfilling, which
resulted in my continued participation in this capacity through my completion of
my undergraduate degree.

The intrinsic rewards of teaching and witnessing student success drove me to
seek other ways I could contribute to the community of students.  As a result, I
accepted the role as president of the student chapter of the Association of
Computing Machinery to spearhead a post-pandemic revival.  Furthermore, I managed to secure
the program lead position for AI-STAARS, a scholarship program aimed at under-represented and economically disadvantaged
minorities in the field of Computer Science.  In addition to preparing and
executing weekly lesson plans, which was familiar to me from my other role as
facilitator, I was tasked to lead a week long winter coding bootcamp and a ten week summer pathways Artificial Intelligence internship.

Through these experiences, I was exposed to a diverse collection of students, each bringing their own unique stories and perspectives.  I
believe this exposure has strengthened my ability to serve a diverse student body, and I look forward to continuing to grow in this
capacity. I am grateful for the opportunity to contribute to the academic success of my peers, and I am excited to meet and support many
more students in the future.
\end{document}