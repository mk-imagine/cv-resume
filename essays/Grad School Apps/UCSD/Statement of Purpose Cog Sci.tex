\documentclass[10pt]{article}
\usepackage[letterpaper]{geometry}
\usepackage{setspace}
\geometry{top=0.9in, bottom=0.9in, left=0.9in, right=0.9in}
\onehalfspacing

\makeatletter
\def\@maketitle{%
  \newpage
  \null
  \vskip 2em%
  \begin{center}%
    {\large
        \begin{tabular}[t]{c}%
        \@author
        \end{tabular}\par}%
  \let \footnote \thanks
    {\LARGE \@title \par}%
    %{\large \@date}%
  \end{center}%
  \par
  \vskip 1.5em}
\makeatother

\author{Mark Kim}
\title{Department of Cognitive Science}

\begin{document}
\maketitle 
My journey from a reluctant student to a passionate researcher has been an unconventional one, marked by unexpected twists and
turns. Growing up, I had parents whose punitive attempts at improving my academic performance led to my deep resentment of scholarly
pursuits. Unfortunately, as a result, I found myself floundering in school, despite possessing a healthy curiosity for all things. I had a
suspicion that I was interested in STEM subjects and could succeed in a STEM field but lacked the self-awareness and personal fortitude to
succeed in school. This continued through early childhood into adulthood leading to a temporary withdrawal from university.

I spent the following years working and eventually owning and managing several businesses. In those years, I learned much about iterative
processes: planning, analysis, implementation, testing, and evaluation. As a business owner, these tools were my livelihood; without
leveraging them consistently, my business would suffer, so this ritual was one that I practiced regularly. Over time, however, I had come
to recognize that my true interests and passions did not align with my businesses and the markets they were operating in. Nevertheless, it
was exactly this entrepreneurial odyssey that helped me develop a broad set of skills and rediscover my past strengths.

Since my re-entry into academia, my goals have not changed significantly, but my methods have. As is common in any journey, I explored
different routes, each contributing to and honing my interests while also preparing me for doctoral study. My early research experience in
Statistics and then Mathematics pedagogy allowed me to recognize amy latent appetite for research. I also quickly understood that a strong
foundation of Mathematics would better prepare me for the type of research that I wanted to perform, which informed my decision to add it
as a second major. In the following years, I aggressively sought out any and all opportunities that might be related to my research
interests: an NSF REU at the University of Houston on affective research; a research engineering internship that investigated clustering of
phishing emails; leading an early exposure to artificial intelligence program for college first year students; and a research project using
large language models for student advising. Upon completing my bachelors degree, I advanced to a Data Science and Artificial Intelligence
masters program to bridge the gap towards a doctoral degree.

As a undergraduate and graduate student researcher at San Francisco State University (SFSU), I have had the opportunity to explore diverse
research areas. During my undergraduate studies, I was also provided the opportunity to work closely with several professors in the fields
of Statistics, Mathematics, and Mathematics Education, which included change-point analysis, graphical models for brain networks, and
remote instruction pedagogy in Mathematics. This work segued into a summer at the University of Houston's NSF funded REU program followed
by another summer working as a research engineering intern at Cofense Inc. These experiences have culminated in my current role in two
projects: supporting early stage computer science undergraduate students; and utilizing large language models (LLMs) for college program
advising to maximize student success. As the program lead for ``AI-STAARS,'' I, under the supervision of Professor Anagha Kulkarni and
Professor Shasta Ihorn, work closely with students with the aim to improve retention and academic achievement in Computer Science through
providing academic support and stimulating students' sense of belonging and identity. Working under Professor Hui Yang on the
``AdvisingGPT'' research team, we are investigating methods to provide automated course equivalency evaluation and personalized academic
advising, which includes approaches such as instruction fine-tuning, retrieval-augmented generation, prompt engineering, and more
traditional machine learning techniques.

Coming from a teaching-oriented, as opposed to a research-oriented, school such as SFSU, the opportunities for participating in research
are sparse, so I have had to be extremely proactive in seeking out and creating my own opportunities. Not being able to find faculty whose
interests or expertise lie in the specific area of research I am most interested in has not deterred me. Instead, I have followed avenues
of study that I can build upon towards my future research and career goals. Between my active engagement with the computer science
community through leading several student organizations, qualitative research in Professor Ihorn's psychology lab, quantitative research
with Professor Yang, and my choice of coursework, I have been consistently and persistently equipping myself for graduate study in
quantitative cognitive science, and I believe that UC San Diego is the perfect school for me to continue this journey.

Studying at SFSU, one of the most ethnically diverse universities in the U.S., has enriched my understanding of different cultures,
perspectives, and experiences. This vibrant and inclusive environment inspired me to contribute to the academic community, much like the
faculty members who mentored me. This caused me to seek out positions that would allow me to contribute to the community of students where
I became a facilitator in supplemental instruction, which progressed into a program lead position for a program called "AI-STAARS." In the
subsequent program, I support and teach under-represented and economically disadvantaged minorities in Computer Science.  In addition to
preparing and executing weekly lesson plans, which was familiar to me from my previous role as a facilitator, I was tasked to lead a week
long winter coding bootcamp and a ten week summer pathways Artificial Intelligence internship.

Building on my experiences in education and programs like "AI-STAARS," my academic journey has deepened my fascination with the
intersection of artificial intelligence and human cognition. While applied AI technologies like ChatGPT and Gemini capture the public's
imagination, their limitations highlight the vast, uncharted territory in understanding the complexities of human thought. This curiosity
drives my desire to explore biologically plausible neural networks, not only as a means to emulate intelligence but also to uncover
insights into the intricate mechanisms of motivation, learning, and decision-making. Such explorations align with burgeoning research areas
like continual and localized learning, both of which offer promising avenues to bridge the gap between artificial and biological systems.
Recent work into continual learning use a combination of multiple techniques to allow artificial neural networks to consolidate synapses to
mitigate forgetting and strengthen connections between contextual information. Similarly, exploring mechanisms of localized learning could
provide insight into how parts of the brain communicate with each other and reinforce neural connections. Investigating neuron
architectures with these features and modeling them would provide insight into the inner workings of the human brain.

Just as Psychology has informed the development of artificial neural networks, I believe that today's state-of-the-art models could provide
insight into behavior, cognition, affect, and decision-making. This could take many forms such as expanding upon recent research examining
the semantics of different emotional or behavioral states captured by embedding models or delving into the latent representations hidden
within different neural networks. Questioning our own understanding and reflecting upon our thoughts is a hallmark of human cognition,
learning, and long-term decision making. Combining all of these separate, yet related subjects, towards modeling meta-cognition and using
these discoveries to better understand the human mind would be my ultimate goal.

A UCSD education would provide me with the ideal environment to delve deeper into the intersection of psychology, data science, and
artificial intelligence. I aspire to develop innovative tools and techniques that can empower individuals to unlock their full potential
and make informed decisions. By pursuing a Ph.D. at UCSD, I believe I can make a lasting impact in the field of Psychology and contribute
to a future where technology serves as a force for good.

I also believe that studying cognition would be greatly enhanced by incorporating behavioral and affective factors. In the context of
Cognitive Behavioral Theory, I would like to examine the link between the capacity for metacognition, behavior, and emotions. This could
take many forms such as expanding upon recent research examining the semantics of different emotional or behavioral states captured by
embedding models or delving into the latent representations hidden within different neural networks.

Researching novel neural networks that more closely mimic biological systems could help uncover some of the mechanisms of higher cognition.
Likewise, creating a model of cognition that closely simulates biological brains could unlock new insight into neurological diseases,
mental illness and wellness, and learning. By pursuing a doctoral degree in the realm of Computational Psychology at UCSD, I will be among
a diverse community of multidisciplinary experts that would be best equipped to assist me with my research interests. It is my hope that
through my journey I will be making incremental discoveries towards unearthing a greater understanding of the interrelationship between
affect, cognition, and behavior.

Through these experiences, I was exposed to a diverse collection of students, each bringing their own unique stories and perspectives. I
believe this exposure has strengthened my ability to serve a diverse student body, and I look forward to continuing to grow in this
capacity. I am grateful for the opportunity to contribute to the academic success of my peers, and I am excited to meet and support many
more students in the future. These experiences have been the cornerstone of my ambition to teach and perform research.
\end{document}