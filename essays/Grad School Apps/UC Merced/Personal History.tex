\documentclass[12pt]{article}
\usepackage[letterpaper]{geometry}
\usepackage{setspace}
\geometry{top=0.9in, bottom=0.9in, left=0.9in, right=0.9in}
\doublespacing

\makeatletter
\def\@maketitle{%
  \newpage
  \null
  \vskip 2em%
  \begin{center}%
    {\large
        \begin{tabular}[t]{c}%
        \@author
        \end{tabular}\par}%
  \let \footnote \thanks
    {\LARGE \@title \par}%
    %{\large \@date}%
  \end{center}%
  \par
  \vskip 1.5em}
\makeatother

\newcommand{\school}{UC Merced}
\newcommand{\departmentprogram}{Cognitive and Information Sciences Program}
\newcommand{\discipline}{Cognitive Science}
\author{Mark Kim}
\title{Personal History}

\begin{document}
\maketitle
% Personal History
% Please describe aspects of your personal background, accomplishments, or achievements that provide context for your desire to pursue
% graduate studies at UC Merced. You may include educational, familial, cultural, economic, or social experiences, challenges, or
% opportunities relevant to your academic journey. Please also address how you might contribute to social or cultural diversity at UC Merced
% and in your academic discipline, and how your studies might help to serve the common good, including underserved or disadvantaged
% populations if applicable. 

My journey from a reluctant student to a passionate researcher has been an unconventional one, marked by unexpected twists and turns. Growing
up, I had parents whose punitive attempts at improving my academic performance led to my deep resentment of scholarly pursuits.
Unfortunately, as a result, I found myself floundering in school, despite possessing a healthy curiosity for all things.
I had a suspicion that I was interested in STEM subjects and could succeed in a STEM field but lacked the self-awareness and personal
fortitude to succeed in school.  This continued through early childhood into adulthood leading to a temporary withdrawal from university.

After withdrawing from school, I spent the following years working and eventually owning and managing several businesses ranging from
retail printing to food service.  In those years, I learned much about iterative processes: planning, analysis, implementation, testing, and
evaluation.  As a business owner, these tools were my livelihood; without leveraging them consistently, my business would suffer, so this
ritual was one that I practiced regularly. Over time, however, I had come to recognize that my true interests and passions did not align
with my businesses and the markets they were operating in. Nevertheless, it was exactly this entrepreneurial odyssey that helped me develop
a broad set of skills and rediscover my past strengths.  The business acumen that I gained over time also made me increasingly cognizant of
the emergence of big data which, in turn, rekindled my natural inquisitiveness and motivated me to enter back into scholastics with a
newfound passion and the personal tools to succeed.

It became evident to me that juxtaposed with big data was an abundance of bad actors willing to use it to exploit our primitive
impulses to subsequently compel us towards decisions and actions that benefit them alone. This led me to wonder if we could leverage these
technologies in constructive ways to improve our lives. My own experience of being adrift from insufficient self-awareness, and my subsequent
return to school, motivated me to learn what inspires such change. Can the inferential tools that are the foundation of big data use be
turned to discern the unknown factors that inform and incite life-(re)defining transformation?  What are the mechanisms of these decisions
and what motivates them?  I want to comprehend how people find their interests and passions and the hidden drivers that move them towards
that end. How does the relationship between affect, behavior, and cognition have on our decisions?  With the recent explosion in available
data and the computational capacity to utilize it, we can now begin to consider the possibility of modeling these complicated systems to
better understand the mind.  Furthermore, I aspire to find out if it was possible to help people affect positive change in their own lives:
promote motivation, discover latent interests, generate healthy curiosity, improve learning, and more.

Since my re-entry into academia, my goals have not changed significantly, but my methods have. As is common in any journey, I explored
different routes, each contributing to and honing my interests while also preparing me for doctoral study.  My early research experience in
Statistics and then Mathematics pedagogy allowed me to recognize amy latent appetite for research.  I also quickly understood that a strong
foundation of Mathematics would better prepare me for the type of research that I wanted to perform, which informed my decision to add it as
a second major.  In the following years, I aggressively sought out any and all opportunities that might be related to my research
interests: an NSF REU at the University of Houston on affective research; a research engineering internship that investigated clustering of
phishing emails; leading an early exposure to artificial intelligence program for college first year students; and a research project using
large language models for student advising.  Upon completing my bachelors degree, I advanced to a Data Science and Artificial Intelligence
masters program to bridge the gap towards a doctoral degree.

Studying at SFSU, one of the most ethnically diverse universities in the U.S., has enriched my understanding of different cultures,
perspectives, and experiences. This vibrant and inclusive environment inspired me to contribute to the academic community, much like the
faculty members who mentored me.  Furthermore, the supplemental instruction program that I so fondly participated in exposed me to the
possibility of contributing to the community that had welcomed me.  Before this, I assumed that my knowledge and past endeavors in the field
of Computer Science and Mathematics were insufficient to be of value to other students.  Nevetheless, my desire to contribute was great, so
I secured a position grading math homework, which not only solidified my understanding but also helped me recognize the impact of supporting
others' learning.

During my stint as a grader, and in response to COVID-19 pandemic lockdowns, I was provided the chance to assist a professor with a research
project doing literature review on remote learning pedagogy.  Both fascinating and eye-opening to me, this study provided me with insight
into challenges of providing a rigorous education amid financial and racial inequities.  This experience made me reflect deeply on how I
would teach students in Math or Computer Science if given the chance.

Shortly after completing my brief encounter with Mathematics education research, I was awarded the opportunity to become a supplemental
instruction facilitator in Calculus and Computer Science.  Armed with my prior experiences, the prospect of applying my newly gained
knowledge thrilled me.  I was determined to apply what I learned and attempt to deliver an experience that not only inspired students, but
also significantly improved their learning outcomes.  My time in this program was deeply fulfilling, which resulted in my continued
participation in this capacity through my completion of my undergraduate degree.

The intrinsic rewards of teaching and witnessing student success drove me to seek other ways I could contribute to the community of
students.  In this search, I found and filled the vacant role of president of the Association of Computing Machinery Student Chapter to
spearhead a post-pandemic revival.  At around the same time, I also managed to secure the program lead position for AI-STAARS, a scholarship
program aimed at under-represented and economically disadvantaged minorities in the field of Computer Science.  In addition to preparing and
executing weekly lesson plans, which was familiar to me from my other role as facilitator, I was tasked to lead a week long winter coding
bootcamp and a ten week summer pathways Artificial Intelligence internship.

Through these experiences, I was exposed to a diverse collection of students, each bringing their own unique stories and perspectives.  I
believe this exposure has strengthened my ability to serve a diverse student body, and I look forward to continuing to grow in this
capacity. I am grateful for the opportunity to contribute to the academic success of my peers, and I am excited to meet and support many
more students in the future.  These experiences have been the cornerstone of my ambition to teach and perform research.
\end{document}