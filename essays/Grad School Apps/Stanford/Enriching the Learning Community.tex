\documentclass[12pt]{article}
\usepackage[letterpaper]{geometry}
\usepackage{setspace}
\geometry{top=0.9in, bottom=0.9in, left=0.9in, right=0.9in}
\doublespacing

\makeatletter
\def\@maketitle{%
  \newpage
  \null
  \vskip 2em%
  \begin{center}%
    {\large
        \begin{tabular}[t]{c}%
        \@author
        \end{tabular}\par}%
  \let \footnote \thanks
    {\LARGE \@title \par}%
    %{\large \@date}%
  \end{center}%
  \par
  \vskip 1.5em}
\makeatother

\author{Mark Kim}
\title{Essay 1}

\begin{document}
\maketitle
% Stanford University welcomes graduate applications from individuals with a broad range of experiences, interests, and backgrounds who would
% contribute to our community of scholars. We invite you to share the lived experiences, demonstrated values, perspectives, and/or activities
% that shape you as a scholar and would help you to make a distinctive contribution to Stanford University. 
% Your statement should not exceed 500 words in length.

When I re-entered college I never expected to have an interest in serving a college or university as a member of the faculty.  My
perspective quickly shifted when I joined a student-led supplemental instruction course in introductory programming, which sparked an
interest in teaching.  Studying at San Francisco State University (SFSU), one of the most ethnically diverse universities in the U.S., has enriched
my understanding of different cultures, perspectives, and experiences. This vibrant, inclusive environment inspired me to contribute to the
academic community, much like the faculty members who mentored me.

In my third year at SFSU, I was awarded the opportunity to become a supplemental instruction (SI) facilitator in Calculus and Computer Science.
Having participated in the SI program as a student in my first year, the prospect of applying my recently acquired knowledge thrilled me.  I
was determined to apply what I learned and attempt to deliver an experience that not only inspired students, but also significantly
improved their learning outcomes.  My time in this program was deeply fulfilling, which resulted in my continued participation in this
capacity through the completion of my undergraduate degree.

The intrinsic rewards of teaching and witnessing student success drove me to seek other ways I could contribute to the community of
students.  As a result, I accepted the role as president of the student chapter of the Association of Computing Machinery (ACM) to spearhead a
post-pandemic revival.  In my role, my goal was singular: to provide as much support and value to all SFSU students interested in computing,
both academically and professionally.
The growth was explosive, beginning from a small core of four student officers, to what is now a large organization of thirty officers,
twelve sub-organizations, and more than 200 active members.  In addition, I believe that, as an organization, we have made a significant
impact on students' well-being, and academic and professional success. Through running ACM, I learned that providing an enthusiastic, inclusive, and
supportive environment, is contagious, which is proven by the fact that ACM has been thriving ever since its revival.

Around the same time as I began reviving ACM, I managed to secure a position as program lead for AI-STAARS, a scholarship program aimed at
under-represented and economically disadvantaged minorities in the field of Computer Science.  In addition to preparing and executing weekly
lesson plans, which was familiar to me from my other role as facilitator, I was tasked to lead a week long winter coding bootcamp and a ten
week summer pathways Artificial Intelligence program.

These experiences have exposed me to a diverse collection of students, each bringing their own unique stories and perspectives.  I
believe this exposure has strengthened my ability to serve a diverse student body, and I look forward to continuing to grow in this
capacity. I am grateful for the opportunity to contribute to the academic success of my peers, and I am eager to contribute these kinds of
values to Sanford University.


\end{document}