\documentclass[12pt]{article}
\usepackage[letterpaper]{geometry}
\usepackage{setspace}
\geometry{top=0.9in, bottom=0.9in, left=0.9in, right=0.9in}
\doublespacing

\makeatletter
\def\@maketitle{%
  \newpage
  \null
  \vskip 2em%
  \begin{center}%
    {\large
        \begin{tabular}[t]{c}%
        \@author
        \end{tabular}\par}%
  \let \footnote \thanks
    {\LARGE \@title \par}%
    %{\large \@date}%
  \end{center}%
  \par
  \vskip 1.5em}
\makeatother

\author{Mark Kim}
\title{Essay 1}

\begin{document}
\maketitle
% Cognitive Science Statement of Purpose

% 2500 word limit.

% Include the following information in your statement:

%     Describe one research problem, project or area for graduate study that excites you
%     How has your background prepared you to pursue such a research problem?

% OPTIONAL (and separate):
% Experiential Questions
% Leadership (e.g., coordination of volunteer activities, board member in student organization, residential life, etc.)
% Overcoming Adversity (e.g., overcoming educational, social, cultural, economic, barriers, or barriers related to accessibility, etc.)
% Community Involvement (e.g., volunteer service, organizing, activism, teaching, mentoring, counseling, community art production, etc.)
% Social Justice Experience (e.g., addressing a systemic inequality through education, organizing, activism, mentorship, counseling, outreach/access, survival and development work, event planning/coordination, community building and development, etc.)
% Personal or Professional Ethics (e.g., experience with an ethical code, conduct seminars, IRB training, etc.)
% Research (e.g., undergraduate research, involvement in McNair or similar programs, independent or group study with a professor or other researcher, research outside of academia, full-time research after college, etc.)
% Other (e.g., any other kind of experience or information that you feel will help add to creating a diverse spectrum of ideas, perspectives and experiences)

\end{document}