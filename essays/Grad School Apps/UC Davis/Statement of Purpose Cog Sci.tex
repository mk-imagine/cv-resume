\documentclass[12pt]{article}
\usepackage[letterpaper]{geometry}
\usepackage{setspace}
\geometry{top=0.9in, bottom=0.9in, left=0.9in, right=0.9in}
\doublespacing

\makeatletter
\def\@maketitle{%
 \newpage
 \null
 \vskip 2em%
 \begin{center}%
  {\large
    \begin{tabular}[t]{c}%
    \@author
    \end{tabular}\par}%
 \let \footnote \thanks
  {\LARGE \@title \par}%
  %{\large \@date}%
 \end{center}%
 \par
 \vskip 1.5em}
\makeatother

\newcommand{\school}{UC Davis}
\newcommand{\abbrschool}{UC Davis}
\newcommand{\departmentprogram}{Psychology Program}
\newcommand{\discipline}{Psychology}
\author{Mark Kim}
\title{Academic Statement}

\begin{document}
As a graduate student researcher at San Francisco State University (SFSU), I have had the opportunity to explore diverse research areas.
During my undergraduate studies, I was also provided the opportunity to work closely with several professors in the fields of Statistics,
Mathematics, and Mathematics Education, which included change-point analysis, graphical models for brain networks, and remote instruction
pedagogy in Mathematics. This work segued into a summer at the University of Houston's NSF funded REU program followed by another summer
working as a research engineering intern at Cofense Inc. These experiences have culminated in my current role in two projects: supporting
early stage computer science undergraduate students; and utilizing large language models (LLMs) for college program advising to maximize
student success. As the program lead for ``AI-STAARS,'' I, under the supervision of Professor Anagha Kulkarni and Professor Shasta Ihorn,
work closely with students with the aim to improve retention and academic achievement in Computer Science through providing academic support
and stimulating students' sense of belonging and identity. Working under Professor Hui Yang on the ``AdvisingGPT'' research team, we are
investigating methods to provide automated course equivalency evaluation and personalized academic advising, which includes approaches such
as instruction fine-tuning, retrieval-augmented generation, prompt engineering, and more traditional machine learning techniques.

Coming from a teaching-oriented, as opposed to a research-oriented, school such as SFSU, the opportunities for participating in research are
sparse, so I have had to be extremely proactive in seeking out and creating my own opportunities. Not being able to find faculty
whose interests or expertise lie in the specific area of research I am most interested in has not deterred me. Instead, I have followed
avenues of study that I can build upon towards my future research and career goals. Between my active engagement with the computer science
community through leading several student organizations, qualitative research in Professor Ihorn's psychology lab, quantitative research
with Professor Yang, and my choice of coursework, I have been consistently and persistently equipping myself for graduate study in
quantitative and computational psychology, and I believe that UC Davis is the perfect school for me to continue this journey.

I am interested in studying cognition through computational methods. Just as Psychology has informed the development of artificial neural
networks, I believe that today's state-of-the-art models could provide insight into behavior, cognition, affect, and decision-making. This
could take many forms such as expanding upon recent research examining the semantics of different emotional or behavioral states captured by
embedding models or delving into the latent representations hidden within different neural networks. Questioning our own understanding and
reflecting upon our thoughts is a hallmark of human cognition, learning, and long-term decision making. Combining all of these separate, yet
related subjects, towards modeling meta-cognition and using these discoveries to better understand the human mind would be my ultimate goal.

A UC Davis education would provide me with the ideal environment to delve deeper into the intersection of psychology, data science, and
artificial intelligence. I aspire to develop innovative tools and techniques that can empower individuals to unlock their full
potential and make informed decisions. By pursuing a Ph.D. at UC Davis, I believe I can make a lasting impact in the field of
Psychology and contribute to a future where technology serves as a force for good.
\end{document}