\documentclass[12pt]{article}
\usepackage[letterpaper]{geometry}
\usepackage{setspace}
\geometry{top=0.9in, bottom=0.9in, left=0.9in, right=0.9in}
\doublespacing

\makeatletter
\def\@maketitle{%
  \newpage
  \null
  \vskip 2em%
  \begin{center}%
    {\large
        \begin{tabular}[t]{c}%
        \@author
        \end{tabular}\par}%
  \let \footnote \thanks
    {\LARGE \@title \par}%
    %{\large \@date}%
  \end{center}%
  \par
  \vskip 1.5em}
\makeatother

\author{Mark Kim}
\title{Essay 1}

\begin{document}
\maketitle
% Statement of Purpose
% Please describe your scholarly and research area(s) of interest, and your future occupational and/or professional plans. Include your
% academic background and experiences that have prepared you for advanced scholarship, and any other information that may help reviewers
% understand your academic goals, and evaluate your preparation and likelihood to achieve those goals in general, and at UC Merced in
% particular. 

As a graduate student researcher at San Francisco State University (SFSU), I have had the opportunity to explore diverse research areas. My
current work under Prof. Hui Yang on the ``AdvisingGPT'' project involves using foundation models for automated course equivalency evaluation and
personalized academic advising. This research focuses on utilizing foundation and embedding models to enhance student success, while
employing techniques such as instruction fine-tuning, retrieval-augmented generation, and prompt engineering. In addition, I am leading
the ``AI-STAARS'' project under Prof. Anagha Kulkarni, where we aim to improve retention and academic achievement in Computer Science through
providing academic support and stimulating students' sense of belonging and identity.  This work has informed the forthcoming
poster presentation ``Metacognition in Computer Science Learning: Perception vs. Reality,'' at the National Association of School
Psychologists Annual Convention.  In previous research roles, I have also delved into topics like clustering and analysis of phishing emails
as a research engineer intern at Cofense Inc., where I significantly improved computational efficiency and data utilization. My work at the
University of Houston's NSF REU program equipped me with skills in multi-threaded algorithm development and exploratory clustering.  During
my undergraduate studies at SFSU, I was also provided the opportunity to work closely with several professors in the fields of
Statistics, Mathematics, and Mathematics Education, which included change-point analysis, graphical models for brain networks, and remote
instruction pedagogy in Mathematics.

Beyond research-related experiences, I was also very active with instructional activities.  These roles included grading, teaching
assistant, facilitator, program liaison, and as an instructor.  I started as a simple grader for Multivariate Calculus and expanded to being
a full teaching assistant before moving on to facilitating my own supplemental instruction courses.  As a facilitator, I was provided the
opportunity to teach supplemental courses in: Calculus, up to and including Multivariate Calculus; Introduction to Programming; Data
Structures; and Programming Methodology.  After completing my undergraduate studies, I was offered a program liaison position, whose role
is to lead and manage the facilitators in the Supplemental Instruction Program.  These roles have refined my ability to communicate complex
concepts effectively to diverse audiences and provide mentorship for facilitators.

State-of-the-art deep learning algorithms today rely on mechanisms that are
biologically implausible.  They depend on gradient back-propagation, which
computes the gradient of an objective function with respect to the weights of a
neural network.  Such back-propagation raises problematic issues that
demonstrate the improbability of such a process in biology.  First,
back-propagation is a purely linear computation, while biological neurons apply
both linear and non-linear operations. Credit assignment, which is the act of
determining the influence that an action taken will have on future rewards, in
such a paradigm, would require precise knowledge of the gradient in both
directions and exact symmetry for the weights.  Futhermore, artificial neurons
communicate by continuous values, while their biological counterparts
communicate through action potentials, which are binary in nature.  

One
area of research that is of particular interest to me is continual learning, which is a
hallmark of human intelligence.  Recent work into continual learning use a
combination of multiple techniques to allow artificial neural networks to
consolidate synapses to mitigate forgetting and strengthen connections between
contextual information.  Nevertheless, the research still relied on
back-propagation of fully connected networks.  Brain synapses are unidirectional
with physically distinct feed-forward and feedback connections. It is also
believed that the brain is capable of localized learning. Investigating neuron
architectures with these features in continual learning is an exciting prospect
of study for me.

Ultimately, however, I would like to investigate the relationship of affect, behavior, and cognition through computational models.

\end{document}