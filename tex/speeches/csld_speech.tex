\documentclass{article}
\usepackage[letterpaper]{geometry}
\usepackage{setspace}
\geometry{top=1in, bottom=1in, left=1in, right=1in}
\onehalfspacing

\begin{document}
Hi everyone.  I hope you all are enjoying the banquet.  I want to take this time
to tell you a little bit about my story with CSME, and more specifically, the SI
program.  How I started, and what it meant to me.

It started with my first semester here at San Francisco State. I'm sitting in
my first computer science class: CSC 210 - Intro to Java.  A
student entered my class, hyping up this one unit companion course
that has no downside and will turn \emph{any} student into an \emph{A}+ student
(or at least that is what it sounded like to me).
I admit, I was skeptical. It seemed too good to be true, kinda like the
promise of six pack abs in just five minutes a day.  And for those of you who
have heard this spiel, don't tell me you didn't think the same thing.

But I thought, "Hey, what do I have to lose, right?" So I signed up -- and to my surprise,
it was anything but a waste of time. In fact, without that SI class, I
doubt I would have passed the parent course, let alone excelled in it.  The
material we covered provided a solid foundation that
set me up for success in my subsequent computer science classes.  It was so good,
that I took another SI course a year later for Physics 230, which, although was
not quite as thoroughly instrumental for my success, still helped me understand
the substance of the class in greater depth. 

But that's not the end of the story, folks.
I am not standing here just because I took a couple of SI classes. No, I am
standing here because I was fortunate enough to become a facilitator myself.
And that's where I gained the greatest benefit from the program.

Now, I've heard that one often learns more as a teacher than as a student,
but I don't think I've ever experienced it as profoundly as I did with the SI
program.  Constructing lesson plans \dots putting together challenging yet
interesting exercises \dots guiding students' through their pain points -- it
all requires an in-depth understanding of the material that far surpasses what a
student needs to succeed in a particular class.  So while I was
teaching, I was learning \dots a \emph{\textbf{lot}}.  Covering the material over the
course of several semesters, with many different students, cemented the
foundational knowledge in my mind in a way that no amount of straight studying
could have.

But being an SI facilitator wasn't just about learning -- it was also deeply rewarding.
Collaborating with other facilitators to design and execute lesson
plans, then seeing it all unfold throughout the semester and then refining our
approach from semester to semester has been immensely gratifying.  And helping others achieve
their goals provides similar, if not more, fulfillment.  As a matter of fact, I attribute much
of my current desire to continue in academia and eventually teach, to the SI program.

But most of all, I have formed so many connections and friendships amongst my
peers, mentors, and professors who have encouraged, supported,
motivated, and challenged me to be the best version of myself. In addition, I
have met some of my closest friends in the SI program: both students and
co-facilitators; some of whom are here in this room right now.  I cherish these
relationships like gold.

From what I've just told you, I suspect you can guess that I have no regrets
about joining the program.  I just wish I had become a facilitator sooner.
Nevertheless, I will always hold my memories with SI close to my heart.

So, to Jessica, the Program Director, thank you for giving me this opportunity
to participate in the program, which has been such a boon to me. And to the SI
program, thank you for not only helping me succeed academically, but also
for introducing me to some of the coolest and nerdiest people I've ever met.
Honestly, my life wouldn't be nearly as fun without you all.
\end{document}