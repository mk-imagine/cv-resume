\documentclass[12pt]{article}
\usepackage[letterpaper]{geometry}
\usepackage{setspace}
\geometry{top=0.85in, bottom=0.85in, left=0.85in, right=0.85in}
\doublespacing

\makeatletter
\def\@maketitle{%
  \newpage
  \null
  \vskip 2em%
  \begin{center}%
    {\large
        \begin{tabular}[t]{c}%
        \@author
        \end{tabular}\par}%
  \let \footnote \thanks
    {\LARGE \@title \par}%
    %{\large \@date}%
  \end{center}%
  \par
  \vskip 1.5em}
\makeatother

\author{Mark Kim}
\title{Statement of Purpose}

\begin{document}
\maketitle
My journey from a reluctant student to a passionate researcher has been an unconventional one, marked by unexpected twists and turns. Growing
up, I had parents whose punitive attempts at improving my academic performance led to my deep resentment of scholarly pursuits.
Unfortunately, as a result, I found myself floundering in school, despite possessing a healthy curiosity for all things.
I had a suspicion that I was interested in STEM subjects and could succeed in a STEM field but lacked the self-awareness and personal
fortitude to succeed in school.  This continued through early childhood into adulthood leading to a temporary withdrawal from university.

After withdrawing from school, I spent the following years working and eventually owning and managing several businesses ranging from
retail printing to food service.  In those years, I learned much about iterative processes: planning, analysis, implementation, testing, and
evaluation.  As a business owner, these tools were my livelihood; without leveraging them consistently, my business would suffer, so this
ritual was one that I practiced regularly. Over time, however, I had come to recognize that my true interests and passions did not align
with my businesses and the markets they were operating in. Nevertheless, it was exactly this entrepreneurial odyssey that helped me develop
a broad set of skills and rediscover my past strengths.  The business acumen that I gained over time also made me increasingly cognizant of
the emergence of big data which, in turn, rekindled my natural inquisitiveness and motivated me to enter back into scholastics with a
newfound passion and the personal tools to succeed.

It became evident to me that juxtaposed with big data was an abundance of bad actors willing to use it to exploit our primitive
impulses to subsequently compel us towards decisions and actions that benefit them alone. This led me to wonder if we could leverage these
technologies in constructive ways to improve our lives. My own experience of being adrift from insufficient self-awareness, and my subsequent
return to school, motivated me to learn what inspires such change. Can the inferential tools that are the foundation of big data use be
turned to discern the unknown factors that inform and incite life-(re)defining transformation?  What are the mechanisms of these decisions
and what motivates them?  I want to comprehend how people find their interests and passions and the hidden drivers that move them towards
that end. How does the relationship between affect, behavior, and cognition have on our decisions?  With the recent explosion in available
data and the computational capacity to utilize it, we can now begin to consider the possibility of modeling these complicated systems to
better understand the mind.  Furthermore, I aspire to find out if it was possible to help people affect positive change in their own lives:
promote motivation, discover latent interests, generate healthy curiosity, improve learning, and more.

Since my re-entry into academia, my goals have not changed significantly, but the means have. As is common in any journey, I explored
different routes, each contributing to and honing my interests while also preparing me for doctoral study.  As a graduate student researcher
at San Francisco State University (SFSU), I have had the opportunity to explore diverse research areas. During my undergraduate studies, I
was also provided the opportunity to work closely with several professors in the fields of Statistics, Mathematics, and Mathematics
Education, which included change-point analysis, graphical models for brain networks, and remote instruction pedagogy in Mathematics.  This
work segued into a summer at the University of Houston's NSF funded REU program followed by another summer working as a research engineering
intern at Cofense Inc.  These experiences have culminated in my current role in two projects: supporting early stage computer science
undergraduate students; and utilizing large language models (LLMs) for college program advising to maximize student success.  As the program lead
for ``AI-STAARS,'' I, under the supervision of Professor Anagha Kulkarni and Professor Shasta Ihorn, work closely with students with the aim
to improve retention and academic achievement in Computer Science through providing academic support and stimulating students' sense of
belonging and identity.  Working under Professor Hui Yang on the ``AdvisingGPT'' research team, we are investigating methods to provide
automated course equivalency evaluation and personalized academic advising, which includes approaches such as instruction fine-tuning,
retrieval-augmented generation, prompt engineering, and more traditional machine learning techniques.

Coming from a teaching-oriented, as opposed to a research-oriented, school such as SFSU, the opportunities for participating in research are
sparse, so I have had to be extremely proactive in seeking out and creating my own opportunities.  Despite not being able to find faculty
whose interests or expertise lie in the specific area of research I am most interested in, I have still been able to discover lines of study
that I can build upon towards my future research and career goals.  Between my active engagement with the computer science community through
leading several student organizations, qualitative research in Professor Ihorn's psychology lab, quantitative research with Professor Yang,
and my choice of coursework, I have been consistently and persistently equipping myself for graduate study in quantitative and computational
psychology, and I believe that Stanford University is the perfect school for me to continue this journey.

Specifically, I am excited about the potential to work with Professor Eichstaedt and contribute to the Computational Psychology and
Well-Being Lab.  My research interests in leveraging LLMs to improve student outcomes in computer science education
align well with the innovative work conducted at Professor Eichstaedt's Computational Psychology and Well-being Lab. Their focus on
utilizing LLMs for next-generation interventions in health and well-being resonates deeply with my desire to empower individuals through
technology. I believe my background in AI and my passion for using technology to improve human well-being make me a strong candidate to
contribute to the lab's important work.

A Stanford education would provide me with the ideal environment to delve deeper into the intersection of psychology, data science, and
artificial intelligence. Ultimately, I aspire to develop innovative tools and techniques that can empower individuals to unlock their full
potential and make informed decisions. By pursuing a Ph.D. at Stanford, I believe I can make a lasting impact on the field of human-centered
AI and contribute to a future where technology serves as a force for good.
\end{document}