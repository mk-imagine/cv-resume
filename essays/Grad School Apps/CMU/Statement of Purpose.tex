\documentclass[10pt]{article}
\usepackage[letterpaper]{geometry}
\usepackage{setspace}
\geometry{top=0.9in, bottom=0.9in, left=0.9in, right=0.9in}
\singlespacing

\makeatletter
\def\@maketitle{%
  \newpage
  \null
  \vskip 2em%
  \begin{center}%
    {\large
        \begin{tabular}[t]{c}%
        \@author
        \end{tabular}\par}%
  \let \footnote \thanks
    {\LARGE \@title \par}%
    %{\large \@date}%
  \end{center}%
  \par
  \vskip 1.5em}
\makeatother

\author{Mark Kim}
\title{Statement of Purpose}

\begin{document}
\maketitle
My journey from a reluctant student to a passionate researcher has been an unconventional one, marked by unexpected twists and
turns. Growing up, I had parents whose punitive attempts at improving my academic performance led to my deep resentment of scholarly
pursuits. Unfortunately, as a result, I found myself floundering in school, despite possessing a healthy curiosity for all things. I had a
suspicion that I was interested in STEM subjects and could succeed in a STEM field but lacked the self-awareness and personal fortitude to
succeed in school. This continued through early childhood into adulthood leading to a temporary withdrawal from university.

I spent the following years working and eventually owning and managing several businesses. In those years, I learned much about iterative
processes: planning, analysis, implementation, testing, and evaluation. As a business owner, these tools were my livelihood; without
leveraging them consistently, my business would suffer, so this ritual was one that I practiced regularly. Over time, however, I had come
to recognize that my true interests and passions did not align with my businesses and the markets they were operating in. Nevertheless, it
was exactly this entrepreneurial odyssey that helped me develop a broad set of skills and rediscover my past strengths.

As a undergraduate and graduate student researcher at San Francisco State University (SFSU), I have had the opportunity to explore diverse
research areas. During my undergraduate studies, I was also provided the opportunity to work closely with several professors in the fields
of Statistics, Mathematics, and Mathematics Education, which included change-point analysis, graphical models for brain networks, and
remote instruction pedagogy in Mathematics. This work segued into a summer at the University of Houston's NSF funded REU program followed
by another summer working as a research engineering intern at Cofense Inc. These experiences have culminated in my current role in two
projects: supporting early stage computer science undergraduate students; and utilizing large language models (LLMs) for college program
advising to maximize student success. As the program lead for ``AI-STAARS,'' I, under the supervision of Professor Anagha Kulkarni and
Professor Shasta Ihorn, work closely with students with the aim to improve retention and academic achievement in Computer Science through
providing academic support and stimulating students' sense of belonging and identity. Working under Professor Hui Yang on the
``AdvisingGPT'' research team, we are investigating methods to provide automated course equivalency evaluation and personalized academic
advising, which includes approaches such as instruction fine-tuning, retrieval-augmented generation, prompt engineering, and more
traditional machine learning techniques.

Coming from a teaching-oriented, as opposed to a research-oriented, school such as SFSU, the opportunities for participating in research are
sparse, so I have had to be extremely proactive in seeking out and creating my own opportunities. Not being able to find faculty
whose interests or expertise lie in the specific area of research I am most interested in has not deterred me. Instead, I have followed
avenues of study that I can build upon towards my future research and career goals. Between my active engagement with the computer science
community through leading several student organizations, qualitative research in Professor Ihorn's psychology lab, quantitative research
with Professor Yang, and my choice of coursework, I have been consistently and persistently equipping myself for graduate study in
computational bases of cognition, and I believe that Carnegie Mellon University (CMU) is the perfect school for me to continue this journey.

Studying at SFSU, one of the most ethnically diverse universities in the U.S., has enriched my understanding of different cultures,
perspectives, and experiences. This vibrant and inclusive environment inspired me to contribute to the academic community, much like the
faculty members who mentored me. This caused me to seek out positions that would allow me to contribute to the community of students where
I became a facilitator in supplemental instruction, which progressed into a program lead position for a program called "AI-STAARS." In the
subsequent program, I support and teach under-represented and economically disadvantaged minorities in Computer Science.  In addition to
preparing and executing weekly lesson plans, which was familiar to me from my previous role as a facilitator, I was tasked to lead a week
long winter coding bootcamp and a ten week summer pathways Artificial Intelligence internship.

Beyond my academic coursework and research, I recognized there existed a void in the computing community at SFSU while the campus began to
reopen following the worst of the pandemic. Recovering from the trauma of that time was difficult for many students, including myself. I
reached out to the faculty advisor for the Association for Computing Machinery Student Chapter on campus and volunteered to lead its
revival. After building a core team of four passionate students, I impressed upon the team that our mission was to provide students with
the kind of academic and professional support that went beyond what the school was furnishing, while building a community of mutual respect
and friendship.  Our initial efforts were aimed at providing skill-building workshops (e.g. coding, technical interview, resume building)
and professional panel discussions.  Once we garnered a consistent and growing membership, we began focusing on creating specialized groups
that would cater to more specific interests such as game development, algorithms, and cyber-security.  We are often credited not for
contributing to a community of students, but building one from the wreckage of the pandemic.  Today, this thriving community of students
has grown into an organization of 15 special interest groups, over 30 officers, and an active membership of over 300 students interested in
computing.

Through these experiences, I was exposed to a diverse collection of students, each bringing their own unique stories and perspectives.
As I have contributed to the community of students at SFSU, my joy of doing so has only grown.  I believe that my involvement in the vibrant
computer science and mathematics community has strengthened my ability to serve a diverse student body, and I look forward to continuing to
grow in this capacity. I am grateful for the opportunity to contribute to the academic success of my fellow students, and I am excited to
meet and support many more students in the future. These profoundly rewarding experiences have been the cornerstone of my ambition to teach
and have informed my research interests.

Building upon the experiences I have mentioned thus far, my fascination in the intersection of artificial intelligence and human cognition
has only grown. While applied AI technologies like ChatGPT and Gemini capture the public's imagination, their limitations highlight the vast
chasm that still exists between human and machine intelligence. My curiosity lives in this space and drives my desire to explore
biologically plausible neural networks, not only as a means to emulate intelligence but also to uncover insights into the intricate
interplay between motivation, learning, and decision-making. Such explorations align with burgeoning research areas like continual and
localized learning, both of which offer promising avenues to bridge the gap between artificial and biological systems. Recent work into
continual learning use a combination of multiple techniques to allow artificial neural networks to consolidate synapses to mitigate
forgetting and strengthen connections between contextual information. Similarly, exploring mechanisms of localized learning could provide
insight into neural network optimization where parts of a neural network are updated from local signals rather than being dependent on the
entire model.

Many breakthroughs in artificial intelligence have deep roots in biomimicry, and I believe that we still have much to learn from the
disciplines of psychology and cognitive science.  In the context of Cognitive Behavioral Theory, I would like to examine the link between
the behavior, emotions, learning, and metacognition. This could take many forms such as expanding upon recent research examining the
semantics of different emotional or behavioral states captured by embedding models or delving into the latent representations hidden within
different neural networks. Questioning our own understanding and reflecting upon our thoughts is a hallmark of human cognition, learning,
and long-term decision making. Combining all of these separate, yet related subjects could unlock new paradigms of artificial intelligence.

My interest in artificial intelligence extends beyond an intellectual interest in computers and computation. In fact, my passion is closely
associated with the cognitive sciences.  Researching novel neural networks that more closely mimic biological systems could help uncover
a greater understanding of the human mind could unlock new insight into neurological diseases, mental illness and wellness, and learning. By
pursuing a doctoral degree in Machine Learning at CMU, I will have the opportunity to apply for the Joint Machine Learning and Neural
Computation program. Leveraging the diverse community of paragons in computer science with the expertise housed in the Neuroscience
Institute would provide a nearly unparalleled resource equipped to assist me with my research interests. It is my hope that through my
journey I will be making incremental discoveries towards unearthing more powerful and effective models of intelligence while broadening our
understanding of the interrelationship between affect, cognition, and behavior.
\end{document}