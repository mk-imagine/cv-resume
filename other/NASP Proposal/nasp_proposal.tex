\documentclass[12pt]{article}
\usepackage{mysty}
\usepackage[backend=biber]{biblatex}

\addbibresource{nasp_proposal.bib}

\title{}
\date{\today}

\begin{document}
\section{Abstract}
Students majoring in computer science often have little to no computing
experience before entering college and this can affect students' confidence and
performance throughout their education~\cite{alvarado}.  To mitigate this
obstacle, we attempt to boost students’ confidence, reduce learning loss, and
provide foundational support by offering a four day online winter term bootcamp
as a bridge between their introductory and intermediate computer science
courses.  This is a qualitative study examining the experiences of Computer
Science students that participated with an emphasis on self-efficacy, the
importance of providing scaffolding to ensure that they stay within the zone of
proximal development~\cite{vygotsky, wood1975, wood1976}.  Although this is a
study of college students, results may translate to secondary schools or other
settings where introductory computing classes are being offered.

\section{Introduction}
With the introduction and rapid growth of artificial intelligence in recent
years, computer science has increasingly become a driver of innovation in the US
economy, but has still not become an integral part of our eduction
system with only 29 states adopting policies that provide CS course access to all
high school students~\cite{code}.  In 2023, there were over 350,000 open
computing jobs nationwide, but only 90,000 students who received a computer
science bachelor's degree~\cite{code}.  Liu, et. al. (2024) suggests
that access to CS courses in secondary school increases the likelihood that
students will declare a CS major by 10\% and completing a CS bachelor degree by
5\%~\cite{liu}.

Providing inter-session 

\printbibliography
\end{document}