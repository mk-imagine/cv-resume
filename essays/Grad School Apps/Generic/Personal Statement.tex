\documentclass[12pt]{article}
\usepackage[letterpaper]{geometry}
\usepackage{setspace}
\geometry{top=0.9in, bottom=0.9in, left=0.9in, right=0.9in}
\doublespacing

\makeatletter
\def\@maketitle{%
  \newpage
  \null
  \vskip 2em%
  \begin{center}%
    {\large
        \begin{tabular}[t]{c}%
        \@author
        \end{tabular}\par}%
  \let \footnote \thanks
    {\LARGE \@title \par}%
    %{\large \@date}%
  \end{center}%
  \par
  \vskip 1.5em}
\makeatother

\title{Personal Statement}
\author{Mark Kim}

\begin{document}
\maketitle

My life story is one that contains many surprising turns which is the basis for
a rich and diverse history that forms my identity.  Growing up, I had parents
whose punitive attempts at improving my academic
performance led to my deep resentment of scholarly pursuits.  Unfortunately, as
a result, I found myself floundering in school despite possessing a healthy
curiosity for all things.  I had a suspicion that I was interested in STEM
subjects and could succeed in a STEM field but lacked the self-awareness and
personal fortitude to succeed in school.  This continued through early childhood
into adulthood and resulted in my withdrawal from university for a time.

After withdrawing from school, I spent the following years working and
eventually owning and managing multiple businesses ranging from retail printing
to food service.  In those years, I learned much about iterative processes:
planning, analysis, implementation, testing, and evaluation.  As a business
owner, these tools were my livelihood; without leveraging them
consistently, my business would suffer, so I performed this perpetual ritual constantly.
Over time, however, I had come to recognize that my true interests and
passions did not align with my businesses and the markets they were operating
in. Nevertheless, it was exactly this entrepreneurial odyssey that helped me
develop a broad set of skills and rediscover my past strengths.  The business
acumen that I gained over time also made me increasingly cognizant of the emergence
of big data which, in turn, rekindled my natural inquisitiveness and motivated
me to enter back into scholastics with a newfound passion and the personal tools
to succeed.

Over time, it became evident to me that juxtaposed with big data was an
abundance of bad actors willing to use it to exploit our primitive impulses to
subsequently compel us towards decisions and actions that benefit them alone.
This led me to wonder if we could leverage these technologies in constructive
ways to improve our lives. My own experience of being adrift from insufficient
self-awareness and my subsequent return to school, motivated me to learn what
inspires such change. Can these tools be used to discern the unknown factors
that inform and incite life-(re)defining transformation?  I wanted to comprehend
how people find their interests and passions and the hidden drivers that move
them towards that end.  Furthermore, I aspired to find out if it was possible to
help people affect positive change in their own lives.

Since my re-entry into academia, my goals have not changed significantly, but my
roadmap has. As is common in any journey, I explored different routes, each
contributing to and honing my interests while also preparing me for doctoral
study.  My early research experience in Statistics and then Mathematics pedagogy
allowed me to recognize a latent appetite for research.  I also quickly understood
that a strong foundation of Mathematics would better prepare me for the type of
research that I wanted to perform, so I added it as a second major to my Computer
Science degree.  In the following years, I aggressively sought out any
and all opportunities that might be related to my research
interests: an NSF REU at the University of Houston on affective
research; a research engineering internship that investigated
clustering of phishing emails; leading an early exposure to artificial
intelligence program for college first year students; and a research project
using large language models for student advising.  Upon completing my bachelors
degree, I advanced to a Data Science and Artificial Intelligence
masters program to bridge the gap towards a doctoral degree.  As I have been exposed to
more research and teaching opportunities, my determination to pursue and succeed
in a doctoral program has only solidified.  Research and teaching has become my
greatest ambition.
\end{document}