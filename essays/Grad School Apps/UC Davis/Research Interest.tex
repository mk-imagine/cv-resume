\documentclass[12pt]{article}
\usepackage[letterpaper]{geometry}
\usepackage{setspace}
\geometry{top=0.9in, bottom=0.9in, left=0.9in, right=0.9in}
\doublespacing

\makeatletter
\def\@maketitle{%
  \newpage
  \null
  \vskip 2em%
  \begin{center}%
    {\large
        \begin{tabular}[t]{c}%
        \@author
        \end{tabular}\par}%
  \let \footnote \thanks
    {\LARGE \@title \par}%
    %{\large \@date}%
  \end{center}%
  \par
  \vskip 1.5em}
\makeatother

\newcommand{\school}{UC Davis}
\newcommand{\abbrschool}{UC Davis}
\newcommand{\departmentprogram}{Psychology Program}
\newcommand{\discipline}{Psychology}
\author{Mark Kim}
\title{Research or Professional Interest}

\begin{document}
\maketitle
The explosion of deep learning has made technologies such as OpenAI's ChatGPT and Google's Gemini household names. The excitement that
follows such attention, however, engenders a narrow-minded focus, constraining what is possible only to what has become available. Despite
their ability to learn to complete complex specialized tasks and mimic human written composition, such models can only give shallow
understanding on how our minds operate. We may be able to infer some properties of low-level brain function from these neural networks, but
insight into higher cognition still eludes us. These higher functions of cognition are what pique my curiosity. In particular, I wish to
delve into the inner workings of motivation, long-term passions, and decision-making in the face of complex scenarios. The methods with
which I would like to accomplish this is through the investigation and modelling of biologically plausible neural networks as a proxy for
the mind. This may include the study of continual and localized learning, and the complex interplay of affect, behavior, and cognition.

Although research towards biologically plausible machine learning has grown significantly, there are many promising avenues of study that
have yet to be explored. This can be done through expanding upon already completed research or investigating new directions. One area of
research that is of personal interest to me is continual learning, which is a hallmark of biological intelligence. Recent work into
continual learning use a combination of multiple techniques to allow artificial neural networks to consolidate synapses to mitigate
forgetting and strengthen connections between contextual information. Similarly, exploring mechanisms of localized learning could provide
insight into how parts of the brain communicate with each other and reinforce neural connections. Investigating neuron architectures with
these features and modeling them would provide insight into the inner workings of the human brain.

I also believe that studying cognition would be greatly enhanced by incorporating behavioral and affective factors. In the context of
Cognitive Behavioral Theory, I would like to examine the link between the capacity for metacognition, behavior, and emotions. This could
take many forms such as expanding upon recent research examining the semantics of different emotional or behavioral states captured by
embedding models or delving into the latent representations hidden within different neural networks.

Researching novel neural networks that more closely mimic biological systems could help uncover some of the mechanisms of higher cognition.
Likewise, creating a model of cognition that closely simulates biological brains could unlock new insight into neurological diseases, mental
illness and wellness, and learning. By pursuing a doctoral degree in the realm of Computational Psychology at UC Davis, I will be among a
diverse community of multidisciplinary experts that would be best equipped to assist me with my research interests. It is my hope that
through my journey I will be making incremental discoveries towards unearthing a greater understanding of the interrelationship between
affect, cognition, and behavior.

\end{document}