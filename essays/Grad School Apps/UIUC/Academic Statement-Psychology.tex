\documentclass[12pt]{article}
\usepackage[letterpaper]{geometry}
\usepackage{setspace}
\geometry{top=0.9in, bottom=0.9in, left=0.9in, right=0.9in}
\doublespacing

\makeatletter
\def\@maketitle{%
  \newpage
  \null
  \vskip 2em%
  \begin{center}%
    {\large
        \begin{tabular}[t]{c}%
        \@author
        \end{tabular}\par}%
  \let \footnote \thanks
    {\LARGE \@title \par}%
    %{\large \@date}%
  \end{center}%
  \par
  \vskip 1.5em}
\makeatother

\newcommand{\school}{University of Illinois Urbana-Champaign (UIUC)}
\newcommand{\abbrschool}{UIUC}
\newcommand{\departmentprogram}{Psychology Program}
\newcommand{\discipline}{Psychology}
\author{Mark Kim}
\title{Academic Statement}

\begin{document}
\maketitle
The explosion of deep learning has made technologies such as OpenAI's ChatGPT and Google's Gemini household names.  The excitement that
follows such attention, however, engenders a narrow-minded focus, constraining what is possible only to what has become available.  Despite
their ability to learn to complete complex specialized tasks and mimic human written composition, such models can only give shallow
understanding on how our minds  operate. We may be able to infer some properties of low-level brain function from these neural networks, but
insight into higher cognition still eludes us.  These higher functions of cognition are what pique my curiosity.  In particular, I wish to
delve into the inner workings of motivation and long-term passions, but studying these things directly is untenable as the factors involved
with any individual's motivations are not only intractably numerous, but also may not be feasibly measurable.  My ultimate goal is to
instead investigate and model biologically plausible neural networks as a proxy of brain function, which would include the study of
continual and localized learning, and the complex interplay of affect, behavior, and cognition.

As a graduate student researcher at San Francisco State University (SFSU), I have had the opportunity to explore diverse research areas.
During my undergraduate studies, I was also provided the opportunity to work closely with several professors in the fields of Statistics,
Mathematics, and Mathematics Education, which included change-point analysis, graphical models for brain networks, and remote instruction
pedagogy in Mathematics.  This work segued into a summer at the University of Houston's NSF funded REU program followed by another summer
working as a research engineering intern at Cofense Inc.  These experiences have culminated in my current role in two projects: supporting
early stage computer science undergraduate students; and utilizing large language models (LLMs) for college program advising to maximize
student success.  As the program lead for ``AI-STAARS,'' I, under the supervision of Professor Anagha Kulkarni and Professor Shasta Ihorn,
work closely with students with the aim to improve retention and academic achievement in Computer Science through providing academic support
and stimulating students' sense of belonging and identity.  Working under Professor Hui Yang on the ``AdvisingGPT'' research team, we are
investigating methods to provide automated course equivalency evaluation and personalized academic advising, which includes approaches such
as instruction fine-tuning, retrieval-augmented generation, prompt engineering, and more traditional machine learning techniques.

Coming from a teaching-oriented, as opposed to a research-oriented, school such as SFSU, the opportunities for participating in research are
sparse, so I have had to be extremely proactive in seeking out and creating my own opportunities.  Not being able to find faculty
whose interests or expertise lie in the specific area of research I am most interested in has not deterred me.  Instead, I have followed
avenues of study that I can build upon towards my future research and career goals.  Between my active engagement with the computer science
community through leading several student organizations, qualitative research in Professor Ihorn's psychology lab, quantitative research
with Professor Yang, and my choice of coursework, I have been consistently and persistently equipping myself for graduate study in
quantitative and computational psychology, and I believe that \school is the perfect school for me to continue this journey.

Although research towards biologically plausible machine learning has grown significantly, there are many promising paths of study that
have yet to be explored.  This can be done through expanding upon already completed research or investigating new directions.  One
area of research that is of interest to me is continual learning, which is a hallmark of biological intelligence.  Recent work
into continual learning use a combination of multiple techniques to allow artificial neural networks to consolidate synapses to mitigate
forgetting and strengthen connections between contextual information.  Additionally, the brain is capable of localized learning.
Exploring neuron architectures with these features and modeling them would provide valuable insight into the inner workings of the
human brain.

I also believe that studying cognition would be greatly enhanced by incorporating behavioral and affective factors.  According to
Cognitive Behavioral Theory, a person's feelings and behavior impacts how they learn.  In the context of Cognitive Behavioral Theory, I would
like to examine the link between the capacity for metacognition, behavior, and emotions.  This could take many forms such as expanding upon
recent research examining the semantics of different emotional or behavioral states captured by embedding models or delving into the latent
representations hidden within different neural networks.  Combining all of these separate, yet related subjects, towards modeling
meta-cognition and using these discoveries to better understand the human mind would be my ultimate goal.  Questioning our own understanding
and reflecting upon our thoughts is a hallmark of human cognition, learning, and long-term decision making.  Gaining ever increasing insight
in how these systems interact and affect each other is my life's ambition and I the diversity of expertise housed within \abbrschool would
provide a strong foundation from which I could further develop my skills. I believe my background with artificial intelligence and teaching,
coupled with my passion to increasing diversity and equity in computing and beyond, make me a strong candidate to contribute to the school's
important work.

\end{document}