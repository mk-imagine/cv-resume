\documentclass[12pt]{article}
\usepackage[letterpaper]{geometry}
\usepackage{setspace}
\geometry{top=0.9in, bottom=0.9in, left=0.9in, right=0.9in}
\doublespacing

\makeatletter
\def\@maketitle{%
  \newpage
  \null
  \vskip 2em%
  \begin{center}%
    {\large
        \begin{tabular}[t]{c}%
        \@author
        \end{tabular}\par}%
  \let \footnote \thanks
    {\LARGE \@title \par}%
    %{\large \@date}%
  \end{center}%
  \par
  \vskip 1.5em}
\makeatother

\newcommand{\school}{University of Illinois Urbana-Champaign (UIUC)}
\newcommand{\abbrschool}{UIUC}
\newcommand{\departmentprogram}{Computer Science Program}
\newcommand{\discipline}{Computer Science}
\author{Mark Kim}
\title{Academic Statement}

\begin{document}
\maketitle
My journey from a reluctant student to a passionate researcher has been an unconventional one, marked by unexpected twists and
turns. Growing up, I had parents whose punitive attempts at improving my academic performance led to my deep resentment of scholarly
pursuits. Unfortunately, as a result, I found myself floundering in school, despite possessing a healthy curiosity for all things. I had a
suspicion that I was interested in STEM subjects and could succeed in a STEM field but lacked the self-awareness and personal fortitude to
succeed in school. This continued through early childhood into adulthood leading to a temporary withdrawal from university.

I spent the following years working and eventually owning and managing several businesses. In those years, I learned much about iterative
processes: planning, analysis, implementation, testing, and evaluation. As a business owner, these tools were my livelihood; without
leveraging them consistently, my business would suffer, so this ritual was one that I practiced regularly. Over time, however, I had come
to recognize that my true interests and passions did not align with my businesses and the markets they were operating in. Nevertheless, it
was exactly this entrepreneurial odyssey that helped me develop a broad set of skills and rediscover my past strengths.

As a graduate student researcher at San Francisco State University (SFSU), I have had the opportunity to explore diverse research areas.
During my undergraduate studies, I was also provided the opportunity to work closely with several professors in the fields of Statistics,
Mathematics, and Mathematics Education, which included change-point analysis, graphical models for brain networks, and remote instruction
pedagogy in Mathematics. This work segued into a summer at the University of Houston's NSF funded REU program followed by another summer
working as a research engineering intern at Cofense Inc. These experiences have culminated in my current role in two projects: supporting
early stage computer science undergraduate students; and utilizing large language models (LLMs) for college program advising to maximize
student success. As the program lead for ``AI-STAARS,'' I, under the supervision of Professor Anagha Kulkarni and Professor Shasta Ihorn,
work closely with students with the aim to improve retention and academic achievement in Computer Science through providing academic support
and stimulating students' sense of belonging and identity. Working under Professor Hui Yang on the ``AdvisingGPT'' research team, we are
investigating methods to provide automated course equivalency evaluation and personalized academic advising, which includes approaches such
as instruction fine-tuning, retrieval-augmented generation, prompt engineering, and more traditional machine learning techniques.

Coming from a teaching-oriented, as opposed to a research-oriented, school such as SFSU, the opportunities for participating in research are
sparse, so I have had to be extremely proactive in seeking out and creating my own opportunities. Not being able to find faculty
whose interests or expertise lie in the specific area of research I am most interested in has not deterred me. Instead, I have followed
avenues of study that I can build upon towards my future research and career goals. Between my active engagement with the computer science
community through leading several student organizations, qualitative research in Professor Ihorn's psychology lab, quantitative research
with Professor Yang, and my choice of coursework, I have been consistently and persistently equipping myself for graduate study in
computational bases of cognition, and I believe that the \school is the perfect school for me to continue this journey.

Building on my experiences in education and programs like "AI-STAARS," my academic journey has deepened my fascination with the
intersection of artificial intelligence and human cognition. While applied AI technologies like ChatGPT and Gemini capture the public's
imagination, their limitations highlight the vast chasm that still exists between human and machine intelligence. My curiosity lives in this
space and drives my desire to explore biologically plausible neural networks, not only as a means to emulate intelligence but also to uncover
insights into the intricate interplay between motivation, learning, and decision-making. Such explorations align with burgeoning research
areas like continual and localized learning, both of which offer promising avenues to bridge the gap between artificial and biological
systems. Recent work into continual learning use a combination of multiple techniques to allow artificial neural networks to consolidate
synapses to mitigate forgetting and strengthen connections between contextual information. Similarly, exploring mechanisms of localized
learning could provide insight into neural network optimization where parts of a neural network are updated from local signals rather than
being dependent on the entire model.

Many breakthroughs in artificial intelligence have deep roots in biomimicry, and I believe that we still have much to learn from the
disciplines of psychology and cognitive science.  In the context of Cognitive Behavioral Theory, I would like to examine the link between
the behavior, emotions, learning, and metacognition. This could take many forms such as expanding upon recent research examining the
semantics of different emotional or behavioral states captured by embedding models or delving into the latent representations hidden within
different neural networks. Questioning our own understanding and reflecting upon our thoughts is a hallmark of human cognition, learning,
and long-term decision making. Combining all of these separate, yet related subjects could unlock new paradigms of artificial intelligence.

My interest in artificial intelligence extends beyond an intellectual interest in computers and computation. In fact, my passion is closely
associated with the cognitive sciences.  Researching novel neural networks that more closely mimic biological systems could help uncover
a greater understanding of the human mind. Creating models of cognition that closely simulates biological brains could unlock new insight
into neurological diseases, mental illness and wellness, and learning.
\end{document}