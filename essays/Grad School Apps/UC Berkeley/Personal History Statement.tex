\documentclass[12pt]{article}
\usepackage[letterpaper]{geometry}
\usepackage{setspace}
\geometry{top=0.9in, bottom=0.9in, left=0.9in, right=0.9in}
\doublespacing

\makeatletter
\def\@maketitle{%
  \newpage
  \null
  \vskip 2em%
  \begin{center}%
    {\large
        \begin{tabular}[t]{c}%
        \@author
        \end{tabular}\par}%
  \let \footnote \thanks
    {\LARGE \@title \par}%
    %{\large \@date}%
  \end{center}%
  \par
  \vskip 1.5em}
\makeatother

\author{Mark Kim}
\title{Personal History Statement}

\begin{document}
\maketitle
% Personal History Statement

% Please describe how your personal background and experiences influenced your decision to pursue a graduate degree. In this section, you may
% also include any relevant information on the following:

%     How you have overcome barriers to access higher education
%     How you have come to understand the barriers faced by others
%     Your academic service to advance equitable access to higher education for women, racial minorities and individuals from other groups
%     that have been historically underrepresented in higher education
%     Your research focusing on underserved populations or related issues of inequality
%     Your leadership among such groups

% The Personal History Statement should not duplicate the Statement of Purpose.
My journey from a reluctant student to a passionate researcher has been an unconventional one, marked by unexpected twists and turns.
Growing up, I had parents whose punitive attempts at improving my academic performance led to my deep resentment of scholarly pursuits.
Unfortunately, as a result, I found myself floundering in school, despite possessing a healthy curiosity for all things. I had a suspicion
that I was interested in STEM subjects and could succeed in a STEM field but lacked the self-awareness and personal fortitude to succeed in
school. This continued through early childhood into adulthood leading to a temporary withdrawal from university.

I spent the following years working and eventually owning and managing several businesses. In those years, I learned much about iterative
processes: planning, analysis, implementation, testing, and evaluation. As a business owner, these tools were my livelihood; without
leveraging them consistently, my business would suffer, so this ritual was one that I practiced regularly. Over time, however, I had come to
recognize that my true interests and passions did not align with my businesses and the markets they were operating in. Nevertheless, it was
exactly this entrepreneurial odyssey that helped me develop a broad set of skills and rediscover my past strengths.

Since my re-entry into academia, my goals have not changed significantly, but my methods have. As is common in any journey, I explored
different routes, each contributing to and honing my interests while also preparing me for doctoral study. My early research experience in
Statistics and then Mathematics pedagogy allowed me to recognize amy latent appetite for research. I also quickly understood that a strong
foundation of Mathematics would better prepare me for the type of research that I wanted to perform, which informed my decision to add it as
a second major. In the following years, I aggressively sought out any and all opportunities that might be related to my research interests:
an NSF REU at the University of Houston on affective research; a research engineering internship that investigated clustering of phishing
emails; leading an early exposure to artificial intelligence program for college first year students; and a research project using large
language models for student advising. Upon completing my bachelors degree, I advanced to a Data Science and Artificial Intelligence masters
program to bridge the gap towards a doctoral degree.

Studying at SFSU, one of the most ethnically diverse universities in the U.S., has enriched my understanding of different cultures,
perspectives, and experiences. This vibrant and inclusive environment inspired me to contribute to the academic community, much like the
faculty members who mentored me. This caused me to seek out positions that would allow me to contribute to the community of students where I
became a facilitator in supplemental instruction, which segued into a program lead position for a program called "AI-STAARS." In the
subsequent program, I support and teach under-represented and economically disadvantaged minorities in Computer Science.  In addition to
preparing and executing weekly lesson plans, skills I had acquired in my previous role as a facilitator, I was tasked to lead a week
long winter coding bootcamp and a ten week summer pathways accelerated Artificial Intelligence internship program.

Beyond my academic coursework and research, I recognized there existed a void in the computing community at SFSU while the campus began to
reopen following the worst of the pandemic. Recovering from the trauma of that time was difficult for many students, including myself. I
reached out to the faculty advisor for the Association for Computing Machinery Student Chapter on campus and volunteered to lead its
revival. After building a core team of four passionate students, I impressed upon the team that our mission was to provide students with
the kind of academic and professional support that went beyond what the school was furnishing, while building a community of mutual respect
and friendship.  Our initial efforts were aimed at providing skill-building workshops (e.g. coding, technical interview, resume building)
and professional panel discussions.  Once we garnered a consistent and growing membership, we began focusing on creating specialized groups
that would cater to more specific interests such as game development, algorithms, and cyber-security.  We are often credited not for
contributing to a community of students, but building one from the wreckage of the pandemic.  Today, this thriving community of students
has grown into an organization of 15 special interest groups, over 30 officers, and an active membership of over 300 students interested in
computing.

Through these experiences, I was exposed to a diverse collection of students, each bringing their own unique stories and perspectives.
As I have contributed to the community of students at SFSU, my joy of doing so has only grown.  I believe that my involvement in the vibrant
computer science and mathematics community has strengthened my ability to serve a diverse student body, and I look forward to continuing to
grow in this capacity. I am grateful for the opportunity to contribute to the academic success of my fellow students, and I am excited to
meet and support many more students in the future. These profoundly rewarding experiences have been the cornerstone of my ambition to teach
and perform research.
\end{document}